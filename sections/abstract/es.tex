\begin{abstract}
  \normalsize
  Las \gls{rnnes} son un tipo de redes neuronales que utilizan conexiones de retroalimentación. Se utilizan varios tipos de modelos de RNN para pronosticar series temporales. Este proyecto se llevó a cabo para estudiarlos teóricamente y desarrollar modelos para predecir el uso de la bicicleta en la ciudad de Chicago basados en el enfoque de las Redes Neuronales Recurrentes (RNN) y para medir la precisión de los modelos desarrollados y comparar los resultados con otras redes neuronales utilizando la técnica del \textit{pass forward} o proyectos similares. En los modelos de construcción se emplearon las técnicas de \textit{Feedforward}, \textit{Simple Recurrent Neural Network} (SRNN), \textit{Long Short Term Memory} (LSTM) y arquitecturas auto-regresivas. Los modelos predicen en base a una ventana corrediza de intervalos de 1 hora. La salida de los modelos es un vector con valores con la predicción de cada estación de la ciudad para un determinado intervalo. La fecha y la hora es la única entrada que la red neuronal y con ella se han calculado resultados sobresalientes. La métrica utilizada para medir la precisión del modelo está en la misma línea que las redes de LSTM y AR de los proyectos de última generación, que generalmente producen menores errores en comparación con las redes de alimentación, pero en algunas ocasiones, el error es mayor que las redes de alimentación, pero depende de las entradas que se espera.
\newline

  \textbf{Palabras clave:} Redes neuronales, redes neuronales recurrentes, predicción, series de tiempo, alquiler de bicicletas.
\end{abstract}