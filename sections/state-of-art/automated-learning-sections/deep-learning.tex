\subsubsection{Aprendizaje profundo}
Las redes neuronales ha sido una de las tecnologías que más se han desarrollado y mejores resultados han aportado en el campo de la Inteligencia Artificial en los últimos años. Lo importante de este tipo de algoritmos es que son capaces de aprender de forma jerarquizada, es decir, la información se aprende por niveles, donde en la primera capa se aprende conceptos muy concretos y en las capas posteriores se usa la información previamente para aprender conceptos más abstractos. Cuanta más capas haya, la información se convierte en más abstractas e interesantes. No existe límite teórico de la cantidad de capas que puede tener una red neuronal y esto da lugar a algoritmos cada vez más complejos y avanzados. Este incremento en el número de capas y en la complejidad es lo que hace que estos algoritmos sean conocidos como algoritmos de Deep Learning (Aprendizaje profundo).
\newline

Uno de los requisitos de estos algoritmos es tener una gran cantidad de datos previos para poder entrenar al modelo y actualmente la sociedad está comenzando a sumergirse en la era de la información con la llegada de la digitalización, abaratamiento de los dispositivos de almacenamiento y un cambio de mentalidad a la hora de apreciar el valor de los datos, gran cantidad de organizaciones han comenzado una tendencia de acumular más y más datos, lo que se ha denominado Big Data.
\newline

Con esa gran cantidad de datos, se necesitan algoritmos de Deep Learning para poder convertir esa gran cantidad de datos en conocimiento. En conclusión, un algoritmo de Deep learning es al fin y al cabo un tipo de red neurona.
\newline