\subsubsection{Algoritmo de \textit{feed-forward}}\label{feedforward}
El algoritmo \textit{feed-forward} es el algoritmo básico con el cual una red neuronal predice $y$. Como se ha visto en la ecuación \ref{eqn:feedforwardexample}, el algoritmo es al fin y al cabo una combinación de productos de matrices y funciones de activaciones. Se puede definir una ecuación general de la siguiente forma:

\begin{equation}
    y = a(W^{L_l} \cdot (a(W^{L_{l-1}} \cdot ... \cdot (a(W^{L_1} \cdot x')^{L_1})) \text{ } ... \text{ } ^{L_{l-1}}))^{L_l}
    \label{eqn:feedforward}
\end{equation}

Esta ecuación está computando el algoritmo \textit{feed-forward}, que se resume en los siguientes pasos:

\begin{enumerate}
\item Se tendrá un vector $x$:
\begin{enumerate}
    \item Si es la primera capa del modelo, $x$ es el dado al modelo como argumento de entrada (vector $x$) pero traspuesto.
    \item E.o.c. $x$ vendrá dado calculado por la capa anterior (vector $y$).
\end{enumerate}
\item Se computa $x'$ introduciendo un $1$ en el inicio del vector.
\begin{equation}
  x' = [1] \oplus x
\end{equation}

\item Se calculará $y$:
\begin{equation}
  y = a(Wx')
\end{equation}
\item Si es la última capa, la predicción del modelo será $y$, e.o.c. volver al paso 1.b.
\end{enumerate}

El mismo algoritmo en pseudocódigo:

\begin{algorithm}[H]
\caption{Algoritmo de \textit{feed-forward}}
\SetAlgoLined
\KwResult{Cálculo vector $y$}
 i = 0\;
 n = Número de capas de la red\;
 x = Vector de entrada\;
 y = Vector de salida\;
 \While{$i\neq n$}{
  \eIf{i == 0}{
   $y' = x^T$\;
   }{
   $y' = y$\;
  }
  $x' = [[1] \oplus y']$\;
  $z = W^{L_i}x'$\;
  $y = a^{L_i}(z)$\;
  $i++$\;
 }
\end{algorithm}


Con esta ecuación es fácil ver que el uso de una función de activación por cada capa es necesaria. Como se ha explicado en la sección \ref{activationfunction}, sin el uso de las funciones de activación, un conjunto de capas puede ser compactada en una sola capa. A continuación, se muestra la demostración matemática:
\begin{equation}
\centering
    \begin{split}
    y = W^{L_l} \cdot W^{L_{l-1}} \cdot ... \cdot W^{1} \cdot x' &= (\prod_{l;i-=1}^{i=1} W^{L_i})x' = W'x'\\
    \text{donde}~W' &= (\prod_{l;i-=1}^{i=1} W^{L_i})
  \end{split}
    \label{eqn:feedforwardnoactivation}
\end{equation}

Esta red colapsa a una única capa cuyos parámetros vienen dados por la matriz de $W'$. Usando el ejemplo de la figura \ref{fig:basic_network51}, si se decidiera no usar funciones de activación, la dimensión de la matriz $W'$ sería de $(3,1)$ que coincide con el tamaño del vector de entrada($x$) y el tamaño del vector de salida($y$).

\begin{equation}
\centering
    \begin{split}
    dim(W^1&) = (3,6) \\
    dim(W^2&) = (6,5) \\
    dim(W^3&) = (5,1) \\ \\
    dim(W^1 &\cdot W^2) = (3,5)\\
    dim(W') &= dim(W^1 \cdot W^2 \cdot W^3) = (3,1)
    \end{split}
\end{equation}

Recordando la siguiente regla para el producto de matrices para un producto de matrices $A \times B = C$:

\begin{equation}
\centering
    \begin{split}
    dim(A) &= (m, n) \\
    dim(B) &= (n, k) \\
    dim(C) &= (m, k)
    \end{split}
    \label{eqn:dimensions}
\end{equation}

\begin{itemize}
    \item La dimensión de la matriz $C$ viene dada por el número de filas de $A$, $m$, y el número de columnas de $B$, $n$.
    \item El número de columnas de $A$($m$) debe coincidir con el número de columnas de $B$($n$)
    , condición que se cumple puesto que se está trabajando con redes neuronales \textit{fully-connected}, todas las neuronas de una capa están conectadas a las neuronas de la siguiente capa.
\end{itemize}

Volviendo a la ecuación \ref{eqn:feedforwardnoactivation} y realizando el mismo cálculo que se ha realizado en la ecuación \ref{eqn:dimensions} sobre la red que se quiera, se puede obtener el tamaño de $y$ que será calculada por el modelo. El número de filas indicará el número de datos que se han predicho y el número de columnas es el número de etiquetas por cada dato.
\newline

En la red mostrada en la figura \ref{fig:basic_network51}, tiene como entrada cinco variables. A modo de ejemplo, se podría usar esta red para crear el modelo del siguiente problema: Predecir el coste de una vivienda en una ciudad dado un conjunto de datos de vivienda en esa misma ciudad. Se quiere predecir el valor del precio a partir de las siguientes variables: número de habitaciones, tipo de vivienda, porcentaje de criminalidad, coordenadas geográficas y metros cuadrados. Esto se puede representar como una tabla donde cada fila es una vivienda.
\newline


\begin{table}[H]
  \centering
  \begin{threeparttable}[t]
       \begin{tabular}{l|llrrrr}
        \toprule
        {} &   \textbf{tipo} & \textbf{coords} & \textbf{gc$^1$} & \textbf{m2$^2$} &  \textbf{h$^3$} & \textbf{precio} \\
        \midrule
        \textbf{1} &       Estudio &  $(40.35717, -3.81053)$ &   $5$ &   $41$ & $2$ &  65.378 \\
        \textbf{2} &   Apartamento &  $(40.53125, -3.72054)$ &   $2$ &   $49$ &  $3$ & 79.519 \\
        \textbf{3} &        Dúplex & $(40.45990, -3.60053)$  &   $5$ &   $59$ &  $3$ & 82.578 \\
        \textbf{4} &   Apartamento &  $(40.43437, -3.74870)$ &   $1$ &   $84$ &  $3$ & 99.701 \\
        \textbf{5} &   Apartamento &  $(40.43646, -3.84071)$ &   $3$ &  $100$ & $5$ & 130.978 \\
        \textbf{6} &         Chalé &  $(40.41922, -3.57909)$ &   $1$ &  $126$ & $6$ & 160.890 \\
        \textbf{7} &   Apartamento &  $(40.47372, -3.58115)$ &   $4$ &   $94$ & $4$ & 215.000 \\
        \textbf{8} &         Ático &  $(40.52186, -3.57051)$ &   $4$ &  $142$ & $6$ & 285.398 \\
        \textbf{9} &         Chalé &  $(40.44979, -3.68003)$ &   $3$ &  $185$ & $7$ & 340.589 \\
        \textbf{10} &       Dúplex & $(40.49347, -3.67969)$ & $2$ &  $209$ & $8$ & 426.000 \\
        \bottomrule
         \end{tabular}
     \begin{tablenotes}
     \item[1] gc: Grado de criminalidad del barrio.
     \item[2] m2: metros cuadrados.
     \item[3] h: Número de habitaciones.
   \end{tablenotes}
    \end{threeparttable}%
  \label{tab:houses}%
\caption{Ejemplo de \textit{dataset} con viviendas}
\end{table}%



% TODO Tabla

%La predicción del valor de una vivienda se calculará a partir de cinco variables de entrada. % Por cada vivienda que se quiera estudiar, se tendrá cinco valores con los que trabajar.
