\subsection{Redes neuronales cerebrales humanas}

Las redes neuronales "artificiales" están inspiradas en el cerebro orgánico de los seres humanos llevado a los ordenadores. No es una perfecta comparación, pero donde más similitudes hay entre una red neuronal del cerebro y una artificial es en las neuronas que lo forman\cite{kinsley}. Las neuronas son las partes más simples de la red neuronal y juntando varias de ellas se pueden formar las redes. En los siguientes párrafos se verá el funcionamiento básico de una neurona y alguno de los procesos del cerebro humano para que se pueda comparar con el funcionamiento de una red neuronal artificial.
\newline

Las neuronas son un tipo específico de célula encargadas de enviar información usando señales eléctricas  denominadas impulsos nerviosos a otras neuronas o a órganos efectores. Su estudio se remonta a 1888, cuando Ramón y Cajal comenzó a postular una teoría conocida como "la doctrina de la neurona" \cite{cajal}. En ella se destaca la concepción de la neurona como una unidad discreta y la ley de la polarización dinámica, modelo capaz de explicar la transmisión unidireccional del impulso nervioso. Posteriormente, publicó múltiples múltiples trabajos, entre otras aportaciones describió la hendidura sináptica, sugiriendo que la comunicación entre neuronas se hacía mediante moléculas químicamente bien definidas, llamadas neurotransmisores, sirva como ejemplo una de las moléculas más conocidas, la acetilcolina\cite{loewi}. Los seres vivos son capaces de reaccionar a las modificaciones del medio. En los seres pluricelulares la captación de estos cambios y la respuesta o falta de la misma es la función del tejido nervioso, formado por neuronas. El ejemplo más fácilmente “entendible” es la respuesta ante un estímulo sea un movimiento producido por una contracción muscular.\cite{robertis}
\newline

Las tareas de una neurona son:
\begin{itemize}
\item Conducción y trasmisión de impulsos nerviosos. El mecanismo más simple de acción nerviosa está representado por el arco reflejo monosináptico, que consiste en un circuito neuronal formado por dos neuronas: Una neurona sensorial posee un receptor en una terminación para recibir el estímulo (en las neuronas más “clásicas” está en las dendritas). La información se propaga a lo largo del axón mediante variaciones transitorias del potencial de reposo. Esto es, un cambio del potencial eléctrico transmembrana celular que, produce una onda de excitación que se inicia en el extremo que percibe y se agota en el final del axón con la liberación del neurotransmisor a la hendidura sináptica. Este neurotransmisor será el que provoque otro potencial eléctrico en la siguiente neurona (que lo captará en sus dendritas) o un efecto si es un órgano efector el que está al otro lado de la hendidura. El potencial eléctrico que se desencadena es del tipo todo o nada en una neurona. La gradación de más o menos intensidad de una respuesta dependerá del reclutamiento de neuronas conseguido por la intensidad y tipo de estímulo y la trasmisión de la información a otras neuronas dependerá de la cantidad de neurotrasmisor liberado que estimulará a una o a muchas neuronas (o a ninguna, una o muchas células efectoras)\cite{noback}.
\item Almacenar información instintiva y adquirida. La adaptación a las funciones especializadas se efectúa por medio de diferentes tipos de prolongaciones \cite{noback}. A diferencia de la memoria que hoy en día tenemos en discos duros, la memoria en el cerebro no se puede guardar como tal en elementos tangibles, es decir, en las neuronas no se guardan los recuerdos. Las memorias son señales. Una memoria a corto plazo es un patrón de señales en el cerebro que ocurren en unas neuronas específicas en el córtex prefrontal. Esa señal comienza como un potencial eléctrico creado por los iones entrando y saliendo de las células en un extremo de la neurona creando una señal eléctrica la cual desencadena una liberación de sustancias químicas que producirá en la siguiente neurona un cambio de su potencial eléctrico en su membrana celular, la trasferencia de información es liberación de una molécula que produce un cambio eléctrico en la siguiente neurona. Básicamente, la electricidad y la química son las responsables de que una señal sea transportada a través de muchas neuronas \cite{brainmemory}.\newline

En resumen, una memoria es una reacción en cadena y como se ha mencionado antes, una neurona está conectada como mínimo a miles de neuronas. Por lo que la cantidad de combinaciones y patrones únicos que se pueden generar a partir de las conexiones entre neuronas son casi infinitas\cite{brainmemory}. 
\newline

Para que se cree un nuevo recuerdo, primero debe de existir un estímulo. Esos estímulos son captados por alguna de las neuronas y envían esa señal a otras neuronas con un patrón en específico. Esa memoria recién creada es totalmente única y, por lo tanto, esa memoria es el conjunto de unas neuronas activadas con una actividad especifica entre ellas. En el futuro, si ese patrón vuelve a vuelve a activarse producirá que vuelva la memoria en forma de recuerdos a la persona. Si se reactiva una y otra vez (o hay estímulos acompañantes, importantes para el sujeto receptor, entre otros detalles) el patrón será almacenado en el área de memoria a largo plazo en el hipocampo. En esta área, las neuronas están más juntas y la señal entre ellas es más fuerte para que los químicos y las señales eléctricas no tengan que viajar tan lejos y con el tiempo esas memorias se arraigan debido a que las memorias se reactivan con mayor frecuencia. Por otro lado, si un recuerdo no se reactiva dado un periodo de tiempo puede que se desvanezca o se matice o se cambie en nuevos recuerdos. \cite{brainmemory}.
\end{itemize}