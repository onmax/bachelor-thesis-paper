\subsubsection{\textit{Predicting station-level hourly demand in a large-scale bike-sharing network: A graph convolutional neural network approach}}

Este estudio propone un modelo de red neural convolucional que puede aprender correlaciones heterogéneas ocultas por pares entre estaciones para predecir la demanda horaria a nivel de estación a gran escala.
\newline

Se exploran dos arquitecturas del modelo GCNN-DDGF: El modelo CNN-DDGF que contiene los bloques de convolución y de avance, y GCNNrec-DDGF contiene además un bloque LSTM para capturar las dependencias temporales en la serie de demanda. Además, se proponen cuatro tipos de modelos GCNN cuyas matrices de adyacencia se basan en diversos datos del sistema de bicicletas compartidas, entre ellos la matriz de distancia espacial (SD), la matriz de demanda (DE), la matriz de duración media del viaje (ATD) y la matriz de correlación de la demanda (DC). Estos seis tipos de modelos de GCNN y otros siete modelos de referencia se construyen y comparan en un conjunto de datos de Citi Bike de la ciudad de Nueva York que incluye 272 estaciones y más de 28 millones de transacciones de 2013 a 2016
\newline

Los resultados muestran que el GCNNrecDDGF tiene el mejor rendimiento en términos del RMSE, el MAE y el coeficiente de determinación (R2), seguido por el GCNNreg-DDGF. Superan a los otros modelos. Se encuentra para capturar alguna información similar a los detalles incrustados en las matrices SD, DE y DC. Y lo que es más importante, también descubre correlaciones heterogéneas ocultas por pares entre estaciones que no son reveladas por ninguna de esas matrices.

 
\begin{table}[H]
\centering
\begin{tabular}{rr}
\toprule
 Métrica & Valor \\
\midrule
 MAE &  1.44 \\
 RMSE &  2.46 \\
 R$^2$ &  0.67 \\
\bottomrule
\end{tabular}
\end{table}

Este proyecto es una muestra de un buen caso de uso para una red neuronal, usando las predicciones para mostrarlos en una herramienta web y de esa forma que los usuarios y la empresa puedan tomar decisiones. En cuanto a lo que es la red neuronal considero que se podría haber hecho mucho mejor porque la red usada es muy simple y los hyperparámetros usados son los por defecto de \textit{sci-kit learn} y los resultados son regulares y mejorables.