\subsubsection{Competición de Kaggle: \textit{Shared bikes demand forecasting}} \label{kaggle_competition}

La plataforma \textit{Kaggle} es una plataforma donde los interesados en Ciencia de Datos pueden compartir blogs, conocimientos y \textit{datasets} entre otros recursos. Además, la propia platafor o empresas que quieren contratar nuevo personal que sea experta en Ciencia de Datos realizan competiciones frecuentemente donde cualquier persona o equipo del mundo puede participar.Es por ello, que es común ver múltiples competiciones activas en \textit{Kaggle} con diferentes problemas realcionados con \textit{Big Data } y sus \textit{datasets}.
\newline

Durante el verano de 2020, tuvo lugar una competición \cite{kaggleCompetition} que trataba de resolver el mismo problema que se ha planteado en este presente trabajo: predecir la cantidad de viajes que habrá desde cada una de las estaciones en una ciudad. En este caso, la competición usó un \textit{dataset} que incluía: ciudad (había multiples ciudades), tiempo en horas, cantidad de bicicletas en una hora, multitud de variables en relación con la meteorología(20 variables) y un calendario laboral asociado.
\newline

La métrica que se usó para medir los modelos que entregaban los participantes fue la $RMSLE$ y los mejores resultados obtenidos fueron los siguientes:

\begin{table}[H]
\centering
\begin{tabular}{rr}
\toprule
 Métrica & RMSLE \\
\midrule
 Guillermo Alejandro Chacon &  0.18721 \\
 Maria Garcia &  0.18973 \\
 Eleanor Manley &  0.19837 \\
 Milan Goetz &  0.20110 \\
 Luisa Runge &  0.20162 \\
 Paula San Roman Bueno &  0.20209 \\
 Diego Cuartas &  0.20498 \\
 Valentina Premoli & 0.20569 \\
\bottomrule
\end{tabular}
\end{table}