\subsection{Inteligencia artifical}

La Inteligencia Artificial pertenece al ámbito de las Ciencias de la Computación que durante la última década ha logrado gran reconocimiento y popularidad. A pesar de ser una disciplina que se ha desarrollado en el último lustro prácticamente, ya ha conseguido dar solución a gran cantidad de problemas que en el pasado se consideraban irresolubles o de gran complejidad. Logros como transcripción de voz a texto \cite{voice2text} o viceversa \cite{text2voice}, automatización de procesos con robots \cite{robots}, conducción automática \cite{automaticdriving}, transferencia de estilo \cite{styletransfer} son algunos de los ejemplos más famosos. Los problemas que intenta solucionar la Inteligencia Artificial abarcan a problemas que afectan a cualquier ámbito de la vida.
\newline

Dar una definición exacta de lo que es la Inteligencia Artificial es algo difícil, ya que es un concepto que depende de la propia definición de inteligencia que hoy en día tiene múltiples interpretaciones. Muchos autores proponen su propia definición de Inteligencia Artificial \cite{haugeland, bellman, charniak, winston, kurzweil, knight, nilsson} y si tomamos todas ellas, podríamos extraer una idea común: la inteligencia artificial es una disciplina de la informática que tiene como objetivo crear entidades que puedan \textbf{imitar} comportamientos inteligentes, desde analizar patrones, clasificar elementos en una imagen, predecir valores hasta reconocer voces, conducir, comercializar en bolsa y muchas entidades más en las que un sistema puede simular un comportamiento inteligente. 
\newline


Principalmente la forma de definir el concepto de Inteligencia Artificial se puede agrupar en alguno de los siguientes enfoques \cite{amodernapproach}:
\begin{enumerate}
\item Sistemas que piensan como humanos \cite{haugeland, bellman}: Las capacidades del sistema deben ser propias de seres humanos. Alan Turing (1950) propuso la Prueba de Turing \cite{turing}, la cual pretende poner a prueba estas capacidades y de ser superada, quedaría demostrado que el sistema tendría un comportamiento similar al de un ser humano. La fiabilidad de esta prueba es cuestionable dado que es fácil crear un sistema que imite el comportamiento humano. Imitar no es tener inteligencia propia. \cite{amodernapproach}
\item Sistemas que actúan como humanos \cite{kurzweil, knight}: La prueba de Turing no tenía en cuenta la persona física sino únicamente su inteligencia. Hanard (1991) propuso la prueba total de Turing\cite{harnad} que tiene en cuenta todo tipo de estímulos externos, habilidades de percepción y manipulación de objetos. O sea, añade nuevos requisitos al sistema: visión por computador y robótica que precisan nuevas capacidades: robótica, procesamiento de lenguaje natural, representación del conocimiento, razonamiento automatizado y aprendizaje automático. Estas seis disciplinas son seis de las disciplinas de la inteligencia artificial que en la actualidad son objeto de estudio como se explicará más adelante. \cite{amodernapproach}
\item Sistemas que piensan racionalmente \cite{charniak, winston}: El sistema trataría de imitar la estructura y procesado de la información a como lo haría un ser humano. Para ello, habría que investigar y tener mayor desarrollo de teoría precisa sobre la mente, para así poder plasmar dicha teoría en un sistema. Esto se podría hacer o bien mediante un estudio introspectivo individual o mediante experimentos psicológicos que sirvan para realizar distintas definiciones y requisitos para que el sistema se pueda programar. Para llevar a cabo una prueba, se facilitarían los mismos datos con una estructura y se medirían los tiempos de reacción. Si la estructura y datos de salida, junto al tiempo de reacción son similares a los de un humano, existe evidencia que algunos de los mecanismos del programa se pueden comparar con los que utiliza un ser humano y, por tanto, considerarse inteligente. \cite{amodernapproach}  
\item Sistemas que actúan racionalmente \cite{poole, nilsson}: Ya no serían sistemas los que se construirían sino agentes. Un agente es algo que razona (agente viene del latín \textit{agere}, hacer). De este agente se espera que actúe con la intención de alcanzar el mejor resultado incluso pudiendo tomar la decisión de inacción, si lo considera oportuno en situaciones de incertidumbre. 
\newline

Se considera que este agente debe ser capaz de realizar inferencias a la hora de establecer qué acciones ayudarán a la consecución del objetivo buscado. Para obtener una correcta inferencia no solo depende de la racionalidad, ya que, en ocasiones, la mejor decisión no será consecuencia de un lento proceso de deliberación sino de un acto reflejo o incluso situaciones que no hay nada correcto que hacer y en las que hay que tomar una decisión. Este enfoque, es el más general y complejo a nivel técnico, porque propone el desarrollo de distintas partes: una parte racional que es común al resto de enfoques y una parte de inferencia lógica. Por otro lado, es más sencillo llevar a cabo pruebas con estos agentes ya que la racionalidad está bien definida matemáticamente, y, por lo tanto, se puede descomponer para diseñar agentes que puedan lograrlo de forma comprobable.  \cite{amodernapproach}
\end{enumerate}


Actualmente, las inteligencias artificiales creadas son lo que se conocen de tipo débil. Estos, son sistemas que pueden superar las capacidades de un humano en alguna tarea. Ahora bien, si se selecciona uno de estos sistemas que sobresalen en un dominio muy específico y se intenta que realice otra tarea, el resultado de esta será mucho peor que el de un ser humano. Esta capacidad de poder hacer múltiples tareas al mismo tiempo es una característica muy codiciada en la actualidad que se sigue investigando en todos los departamentos de Inteligencia Artificial. Por el momento, solo existen Inteligencias Artificiales débiles, capaces de hacer extraordinariamente una o varios grupos de tareas. Por el contrario, los sistemas con Inteligencia Artificial fuertes son sistemas que pueden llevar a cabo tareas de distintos dominios, pero aún no existe ejemplo de este tipo de sistemas \cite{amodernapproach}.
\newline

Es importante mencionar que algunos de los cuatro enfoques sobre la definición de Inteligencia Artificial explicados anteriormente tienen como objetivo imitar comportamientos inteligentes. Tratar de imitar un comportamiento puede ser relativamente fácil. Por ejemplo, se puede programar a un sistema para que juegue al ajedrez siguiendo una serie de reglas y predicados. Al ser una máquina podrá haber memorizado partidas similares que hubo en el pasado y saber cuales fueron las estrategias que tomó su contricante y de ese modo tener acceso a mucha más información que su rival. De ese modo, el sistema podrá escoger la mejor decisión en cada movimiento y será imbatible, pero si tratamos de que intente jugar con alguna estrategia que no haya visto antes, no será capaz de usar una estrategia que tenga válida tacticamente hablando, es decir, no funcionará para otro casos porque el sistema solo está imitando y no aprendiendo.
\newline

Imitar no significa que dicho comportamiento sea en esencia un comportamiento cognitivo, sin embargo, según la definición que se ha dado de inteligencia Artificial, el sistema que juega al ajedrez se consideraría un sistema con Inteligencia Artificial, aunque las mecánicas del movimiento se haya memorizado. Es por ello por lo que dentro de este campo de la informática podemos encontrarnos diferentes subcategorías que responden a diferentes comportamientos inteligentes. Estas subcategorías son:
\begin{itemize}
\item Robótica
\item Visión por computador
\item Procesamiento de lenguaje natural 
\item Representación del conocimiento
\item Razonamiento automático
\item Aprendizaje automático
\end{itemize}

Por encima de todo, si hay una categoría que nos define como agentes inteligentes es la capacidad de aprender. Esta capacidad es la que se centra este trabajo y es objeto de estudio del aprendizaje automático que vamos a ver en más detalle a continuación.
\newline