\subsection{Aprendizaje automático}

El aprendizaje automático (\textit{machine learning} en inglés) es la rama de la inteligencia artificial que estudia como dotar a las máquinas de capacidad de aprendizaje entendiendo a éste como la generalización del conocimiento a partir de un conjunto de experiencias. Este aprendizaje puede dividirse en \cite{amodernapproach}:
\begin{itemize}
\item Supervisado
\item No supervisado
\item Reforzado
\end{itemize}
El aprendizaje automático es la rama principal de la Inteligencia Artificial, ya que el resto de las categorías o bien imitan comportamientos o bien aprenden de experiencias, es decir, usan \textit{machine learning} para realizar la tarea que se les encomienda. Una cosa es programar una máquina para que pueda moverse y otra cosa es que la máquina aprenda a moverse. Igualmente, no es lo mismo programar que componentes forman una cara de una persona que automáticamente aprender que es una cara. Este cambio de paradigma es lo que diferencia el \textit{machine learning} de la Inteligencia Artificial, y es por ello por lo que no se debe pensar que son los mismos conceptos.
\newline

El aprendizaje automático es el estudio sistemático de algoritmos y sistemas que mejoran su conocimiento o desempeño con experiencia. En esta subcategoría existen diferentes tipos de aplicaciones siendo algunos de ellos \cite{amodernapproach}:
\begin{itemize}
\item Modelos de regresión: Son modelos que predicen el valor de una o más variables dado un vector de entrada. Una función lineal es el modelo de regresión más simple, pero los modelos de regresión más complejos están formados por un conjunto fijo de funciones no lineales, conocidas como funciones base \cite{flach}. Algoritmos como SVM o las redes neuronal se basan en este tipo de modelos. Este último, será el usado en el presente trabajo.
\newline

\item Árboles de decisión
\newline

\item Modelos de clasificación
\newline
\item Técnicas de agrupación
\newline
\end{itemize}

\subsubsection{Modelos}\label{models}
El universo que se conoce está en constante evolución y es complejo, caótico y enigmático, sin embargo, la inteligencia del ser humano consigue dar sentido a todo ese caos en una búsqueda por la elegancia y la simetría que se esconde entre los patrones que identifica en nuestra realidad. La habilidad de detectar patrones y usarlo para nuestro bien ha sido una de las principales razones para el desarrollo de la especie humana. La ciencia nos ha permitido entender, observar y simplificar el mundo convirtiendo todo este enigma en conocimiento, es decir, reconstruyendo la realidad a través de modelos \cite{marlie}.
\newline

Un modelo es una construcción conceptual y simplificada de una realidad compleja permitiendo entender mejor dicha realidad. Existen multitud de modelos que usamos diariamente, por ejemplo, un mapa. Un mapa nos permite reflejar un mundo tridimensional en una superficie bidimensional eliminando información que no necesitamos procesar como puede ser artefactos del entorno o tipo de vegetación. Otro ejemplo, es una ecuación física, donde se relacionan distintas constantes y valores y de ese modo poder aproximar el comportamiento físico de la realidad. Una partitura también es otro ejemplo de modelo. En este se refleja información de como distintos instrumentos deben coordinarse para poder producir siempre la misma canción. Se podría usar el espectro de frecuencias de la canción para poder representar mejor la realidad, pero esto obviamente, es mucho más complejo de interpretar siendo un ser humano. En resumen, un modelo busca el equilibrio entre representar correctamente la realidad y ser simple para que pueda ser usado \cite{kuhne}.
\newline

Imaginemos que se quiere modelar la meteorología. Para ello se recopilan diferentes evidencias y tras observar se pueden enunciar un primer modelo:

\begin{displayquote}
“El verano será soleado, caluroso y despejado. El invierno será frío, nublado y lloverá.”
\end{displayquote}

Si se sigue recopilando evidencias pronto uno se dará cuenta de que este modelo es muy simple ya que habrá días que hará frío en verano y calor en invierno. Habrá tormentas en verano y días despejados en invierno. Esto provocará mejoras en el modelo en cada iteración. 
\newline

Pero si se sigue estudiando otras evidencias en algún trópico o en algún polo, por ejemplo, se llegará a la conclusión de que el modelo era muy simple y por lo tanto, se tendrá que añadir más reglas para que este modelo se asemeje mejor a la realidad en cualquier parte del mundo.
\newline

Al final, se obtendrá un modelo muy complejo con todas las excepciones y condiciones. Una alternativa a esto es hacer uso de la probabilidad para poder decir matemáticamente que la mayoría de las veces un día de verano será soleado y no un modelo tan complejo del que depender. 
\newline

La probabilidad es la herramienta perfecta para acotar la incertidumbre sobre un tema por falta de conocimientos o datos o para evitar el trabajo de realizar un modelo complejo que conllevaría menor comprensión para la mente humana y dispersión en su aproximación. Poder usar una probabilidad es mucho más fácil que tener que estudiar todas las condiciones físicas de un entorno y el comportamiento de sus entidades para poder saber a ciencia cierta lo que va a ocurrir. Utilizar la probabilidad para construir modelos da como resultado los modelos probabilísticos. Estos modelos comprimen en base a probabilidades muchas de la variabilidad de nuestra realidad siendo más sencillo de gestionar la información que recibimos del entorno \cite{kuhne}.
\newline

Nuestro cerebro aplica esquemas similares a estos modelos probabilísticos y gracias a ellos es que tenemos capacidades como la de conceptualizar, predecir, generalizar, razonar o aprender. Por esto mismo, descubrir cuales son estos modelos es uno de los objetivos básicos del campo del \textit{machine learning} y una de las herramientas fundamentales de la Inteligencia Artificial.
\newline
