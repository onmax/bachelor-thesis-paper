\begin{abstract}
  \normalsize
  Las redes neuronales recurrentes (RNN en inglés) son un tipo de redes neuronales que utilizan conexiones de retroalimentación. Existen diferentes aplicaciones para este tipo de redes neuronales como por ejemplo el estudio de la demanda de un activo durante un tiempo. Este proyecto se llevó a cabo para estudiarlas teóricamente y desarrollar modelos para predecir la demanda de bicicletas en la ciudad de Chicago basados en el enfoque de las RNN y comparar distintas arquitecturas. Los modelos empleados han sido redes neuronales Feedforward, Simple Recurrent Neural Network, LSTM y arquitecturas auto-regresivas. Los modelos predicen en base a una ventana deslizante de intervalos de 1 hora. La salida de los modelos es un vector con valores con la predicción de cada estación de la ciudad para un determinado intervalo. La fecha y la hora es la única entrada que la red neuronal y con ella se han obtenido resultados sobresalientes. Los resultados obtenidos con las redes LSTM y autoregresivas han obtenido resultados que compiten con proyectos de última generación, a pesar de que se han encontrado varios puntos de fallo y posibles mejoras.
\newline

  \textbf{Palabras clave:} Redes neuronales, redes neuronales recurrentes, predicción, series de tiempo, alquiler de bicicletas.
\end{abstract}
