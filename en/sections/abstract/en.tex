\begin{otherlanguage}{english}
  \begin{abstract}
    \normalsize
    Recurrent neural networks (RNN) are a type of neural network that uses feedback connections. There are different applications for this type of neural network, such as studying the demand for an asset over time. This project was carried out to study them theoretically and develop models to predict the demand for bicycles in the city of Chicago based on the RNN approach and to compare different architectures. The models used were Feedforward Neural Networks, Simple Recurrent Neural Network, LSTM and auto-regressive architectures. The models predict on the basis of a sliding window of 1 hour intervals. The output of the models is a vector with values with the prediction of each station in the city for a given interval. The date and time is the only input that the neural network and with it outstanding results have been calculated. The results obtained with the LSTM and autoregressive networks have obtained results that compete with state-of-the-art projects, despite the fact that several points of failure and possible improvements have been found.
\newline

    \textbf{Keywords:} Neural network, recurrent neural network, forecasting, time series, bike sharing.
  \end{abstract}
\end{otherlanguage}