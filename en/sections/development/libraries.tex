\subsection{Used tools and libraries} \label{used_libs}

The language used for the development of this library has been Python 3.6 \cite{python}. This language is a default option together with $R$ in the \acrlong{ai} field. With Python, it has been used a set of libraries that make this language the favourite option to develop this kind of projects. The libraries used in this project have been:
 
\begin{itemize}
    \item Tensorflow \cite{tensorflow2015-whitepaper}: it one of the most important \acrlong{dl} platform nowadays. This Google-based library goes beyond \acrlong{ai}, but its flexibility and large community of developers has positioned it as the leading tool in the \acrlong{dl} sector.
    \item Keras \cite{keras}: Keras is an open source library written in Python, which is mainly based on the work of François Chollet, a Google developer, in the framework of the ONEIROS project. The aim of the library is to accelerate the creation of neural networks: to this end, Keras does not function as a stand-alone framework, but as an intuitive user interface (API) that allows access to and development of various self-learning frameworks. Keras-compatible frameworks include TensorFlow.
    \item Numpy \cite{numpy}: Numpy is the premier library for scientific computing in Python. It has many integrated functions of N-dimensional matrix calculation, as well as the Fourier transform, multiple functions of linear algebra and several randomness functions.
    \item Pandas \cite{pandas}: It is an open source library providing high-performance, easy-to-use data structures and data analysis tools for the Python programming language.
    \item Matplotlib \cite{matplotlib}: It is a library for the generation of graphics from data contained in lists or arrays in the Python programming language and its mathematical extension NumPy. It provides an API, pylab, designed to remind to MATLAB. The graphic theme called SciencePlots \cite{SciencePlots} has been used.
\end{itemize}