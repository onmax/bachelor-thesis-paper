\subsection{Human brain neural networks}


The "artificial" neural networks are inspired by the organic brain of human beings brought to the computers. This is not a perfect comparison, but where there are most similarities between a neural network in the brain and an artificial one is in the neurons that constitute it \cite{kinsley}. Neurons are the simplest parts of the neural network, and by bringing several of them together, networks can be formed. In the following paragraphs we will look at the basic functioning of a neuron and some of the processes of the human brain so that it can be compared with the functioning of an artificial neuron network.
\newline



Neurons are a specific type of cell responsible for sending information using electrical signals called nerve impulses to other neurons or to effector organs. Their study dates back to 1888, when \textit{Ramón y Cajal} began to postulate a theory known as the "neuron doctrine"\cite{cajal}. In it, the concept of the neuron as a discrete unit and the law of dynamic polarisation, a model capable of explaining the unidirectional transmission of the nerve impulse, are highlighted. He later published many papers, among other contributions he described the synaptic cleft, suggesting that communication between neurons was made by chemically well-defined molecules, called neurotransmitters, using as an example one of the best-known molecules, acetylcholine \cite{loewi}. Living beings can react to changes in the environment. In multicellular beings the capture of these changes and the response or lack of it is the function of the nervous tissue, formed by neurons. The most easily "understood" example is the response to a stimulus, be it a movement produced by a muscular contraction \cite{robertis}.
\newline

The tasks of a neuron are:
\begin{itemize}
\item Conduction and transmission of nervous impulses. The simplest mechanism of nerve action is represented by the monosynaptic reflex arc, which consists of a neuronal circuit formed by two neurons: A sensory neuron has a receptor at one end to receive the stimulus (in the more "classical" neurons it is in the dendrites). The information is propagated along the axon by transient variations of the resting potential. This is a change in the electrical transmembrane potential of the cell, which produces an excitation wave that starts at the sensing end and is exhausted at the end of the axon with the release of the neurotransmitter into the synaptic cleft. This neurotransmitter will be the one that causes another electrical potential in the next neuron (which will capture it in its dendrites) or an effect if it is an effector organ that is on the other side of the cleft. The electrical potential that is triggered is of all or nothing type in a neuron. The gradation of intensity of a response will depend on the recruitment of neurons achieved by the intensity and type of stimulus and the transmission of information to other neurons will depend on the amount of neurotransmitter released that will stimulate one or many neurons (or none, one or many effector cells) \cite{noback}.

\item Store instinctive and acquired information. Adaptation to specialised functions is achieved by means of different types of extensions \cite{noback}. Unlike the memory that we have today on hard disks, the memory in the brain cannot be stored as such in tangible elements, i.e., memories are not stored in the neurons. Memories are signals. Short-term memory is a pattern of signals in the brain that occur in specific neurons in the prefrontal cortex. That signal starts as an electrical potential created by ions entering and leaving the cells at one end of the neuron creating an electrical signal which triggers a release of chemicals that will produce in the next neuron a change of its electrical potential in its cell membrane, the transfer of information is a release of a molecule that produces an electrical change in the next neuron. Basically, electricity and chemistry are responsible for a signal being transported through many neurons \cite{brainmemory}.
\newline

In short, a memory is a chain reaction and as mentioned above, one neuron is connected to at least thousands of other neurons. So, the number of unique combinations and patterns that can be generated from connections between neurons is almost infinite \cite{brainmemory}.
\newline

For a new memory to be created, there must first be a stimulus. These stimuli are captured by one of the neurons and send that signal to other neurons with a specific pattern. This newly created memory is totally unique and, therefore, this memory is the set of neurons activated with a specific activity between them. In the future, if that pattern is activated again it will cause the memory to return in the form of memories to the person. If it is reactivated again and again (or there are accompanying stimuli, important for the receiving subject, among other details) the pattern will be stored in the long-term memory area in the hippocampus. In this area, neurons are closer together and the signal between them is stronger so that chemicals and electrical signals do not have to travel as far and over time those memories become more entrenched as memories are reactivated more frequently. On the other hand, if a memory is not reactivated given a period of time it may fade or become nuanced or changed into new memories \cite{brainmemory}.

\end{itemize}