\subsection{Machine learning}

\acrfull{ml} is the branch of \acrshort{ai} that studies how to give machines the ability to learn, understanding it as the generalisation of knowledge from a set of experiences. This learning can be divided into \cite{amodernapproach}:
\begin{itemize}
\item Supervised: The algorithm learns on a labeled dataset, providing an answer key that the algorithm can use to evaluate its accuracy on training data \cite{nvidia}.
\item Unsupervised: Given the unlabeled data, the algorithm tries to capture meaning by extracting characteristics and patterns by itself \cite{nvidia}.
\item Reinforced: An algorithm is trained with a system of rewards, providing feedback when an artificial intelligence agent performs the best action in a particular situation \cite{nvidia}.
\end{itemize}


\acrlong{ml} is the main branch of \acrshort{ai}, since the other branches either imitate behaviour or learn from experience, i.e., they use \acrshort{ml} to perform the task at hand. It is one thing to program a machine so that it can move, but it is another thing for the machine to learn to move. Likewise, it is not the same to program which components form a face of a person as it is to automatically learn that it is a face. This paradigm shift is what differentiates the \acrshort{ml} from the \acrshort{ai}, and that is why it should not be thought that they are the same concepts.
\newline


\acrlong{ml} is the systematic study of algorithms and systems that improve your knowledge or experienced performance. In this subcategory there are different types of applications, some of them \cite{amodernapproach}:
\begin{itemize}
\item Regression models: These are models that predict the value of one or more variables given an input vector. A linear function is the simplest regression model, but the more complex regression models are formed by a fixed set of non-linear functions, known as base functions \cite{flach}. Algorithms such as the acrfull{svm} or the \acrfull{nn} are based on this type of model. The latter will be used in this paper.

\item Decision trees

\item Classification models
\item Grouping techniques
\end{itemize}

\subsubsection{Modelos}\label{models}
El universo que se conoce está en constante evolución y es complejo, caótico y enigmático, sin embargo, la inteligencia del ser humano consigue dar sentido a todo ese caos en una búsqueda por la elegancia y la simetría que se esconde entre los patrones que identifica en nuestra realidad. La habilidad de detectar patrones y usarlo para nuestro bien ha sido una de las principales razones para el desarrollo de la especie humana. La ciencia nos ha permitido entender, observar y simplificar el mundo convirtiendo todo este enigma en conocimiento, es decir, reconstruyendo la realidad a través de modelos \cite{marlie}.
\newline

Un modelo es una construcción conceptual y simplificada de una realidad compleja permitiendo entender mejor dicha realidad. Existen multitud de modelos que usamos diariamente, por ejemplo, un mapa. Un mapa nos permite reflejar un mundo tridimensional en una superficie bidimensional eliminando información que no necesitamos procesar como puede ser artefactos del entorno o tipo de vegetación. Otro ejemplo, es una ecuación física, donde se relacionan distintas constantes y valores y de ese modo poder aproximar el comportamiento físico de la realidad. Una partitura también es otro ejemplo de modelo. En este se refleja información de como distintos instrumentos deben coordinarse para poder producir siempre la misma canción. Se podría usar el espectro de frecuencias de la canción para poder representar mejor la realidad, pero esto obviamente, es mucho más complejo de interpretar siendo un ser humano. En resumen, un modelo busca el equilibrio entre representar correctamente la realidad y ser simple para que pueda ser usado \cite{kuhne}.
\newline

Imaginemos que se quiere modelar la meteorología. Para ello se recopilan diferentes evidencias y tras observar se pueden enunciar un primer modelo:

\begin{displayquote}
“El verano será soleado, caluroso y despejado. El invierno será frío, nublado y lloverá.”
\end{displayquote}

Si se sigue recopilando evidencias pronto uno se dará cuenta de que este modelo es muy simple ya que habrá días que hará frío en verano y calor en invierno. Habrá tormentas en verano y días despejados en invierno. Esto provocará mejoras en el modelo en cada iteración. 
\newline

Pero si se sigue estudiando otras evidencias en algún trópico o en algún polo, por ejemplo, se llegará a la conclusión de que el modelo era muy simple y por lo tanto, se tendrá que añadir más reglas para que este modelo se asemeje mejor a la realidad en cualquier parte del mundo.
\newline

Al final, se obtendrá un modelo muy complejo con todas las excepciones y condiciones. Una alternativa a esto es hacer uso de la probabilidad para poder decir matemáticamente que la mayoría de las veces un día de verano será soleado y no un modelo tan complejo del que depender. 
\newline

La probabilidad es la herramienta perfecta para acotar la incertidumbre sobre un tema por falta de conocimientos o datos o para evitar el trabajo de realizar un modelo complejo que conllevaría menor comprensión para la mente humana y dispersión en su aproximación. Poder usar una probabilidad es mucho más fácil que tener que estudiar todas las condiciones físicas de un entorno y el comportamiento de sus entidades para poder saber a ciencia cierta lo que va a ocurrir. Utilizar la probabilidad para construir modelos da como resultado los modelos probabilísticos. Estos modelos comprimen en base a probabilidades muchas de la variabilidad de nuestra realidad siendo más sencillo de gestionar la información que recibimos del entorno \cite{kuhne}.
\newline

Nuestro cerebro aplica esquemas similares a estos modelos probabilísticos y gracias a ellos es que tenemos capacidades como la de conceptualizar, predecir, generalizar, razonar o aprender. Por esto mismo, descubrir cuales son estos modelos es uno de los objetivos básicos del campo del \textit{machine learning} y una de las herramientas fundamentales de la Inteligencia Artificial.
\newline
