\subsection{Machine learning}

\acrfull{ml} is the branch of \acrshort{ai} that studies how to give machines the ability to learn, understanding it as the generalisation of knowledge from a set of experiences. This learning can be divided into \cite{amodernapproach}:
\begin{itemize}
\item Supervised: The algorithm learns on a labeled dataset, providing an answer key that the algorithm can use to evaluate its accuracy on training data \cite{nvidia}.
\item Unsupervised: Given the unlabeled data, the algorithm tries to capture meaning by extracting characteristics and patterns by itself \cite{nvidia}.
\item Reinforced: An algorithm is trained with a system of rewards, providing feedback when an artificial intelligence agent performs the best action in a particular situation \cite{nvidia}.
\end{itemize}


\acrlong{ml} is the main branch of \acrshort{ai}, since the other branches either imitate behaviour or learn from experience, i.e., they use \acrshort{ml} to perform the task at hand. It is one thing to program a machine so that it can move, but it is another thing for the machine to learn to move. Likewise, it is not the same to program which components form a face of a person as it is to automatically learn that it is a face. This paradigm shift is what differentiates the \acrshort{ml} from the \acrshort{ai}, and that is why it should not be thought that they are the same concepts.
\newline


\acrlong{ml} is the systematic study of algorithms and systems that improve your knowledge or experienced performance. In this subcategory there are different types of applications, some of them \cite{amodernapproach}:
\begin{itemize}
\item Regression models: These are models that predict the value of one or more variables given an input vector. A linear function is the simplest regression model, but the more complex regression models are formed by a fixed set of non-linear functions, known as base functions \cite{flach}. Algorithms such as the acrfull{svm} or the \acrfull{nn} are based on this type of model. The latter will be used in this paper.

\item Decision trees

\item Classification models
\item Grouping techniques
\end{itemize}

\subsubsection{Models}\label{models}

The universe we know is in constant evolution and is complex, chaotic and enigmatic, however, the intelligence of the human being manages to give meaning to all that chaos in a search for the elegance and symmetry that is hidden among the patterns it identifies in our reality. The ability to detect patterns and use them for our own good has been one of the main reasons for the development of the human species. Science has allowed us to understand, observe and simplify the world by turning all this enigma into knowledge, that is, by reconstructing reality through models \cite{marlie}.
\newline

A model is a conceptual and simplified construction of a complex reality allowing a better understanding of that reality. There are many models that we use every day, for example, a map. A map allows us to reflect a three-dimensional world on a two-dimensional surface, eliminating information that we do not need to process, such as environmental artifacts or types of vegetation. Another example is a physical equation, where different constants and values are related and in this way, we can approximate the physical behaviour of reality. A score is another example of a model. It reflects information on how different instruments must be coordinated in order to always produce the same song. The frequency spectrum of the song could be used to better represent reality, but this is obviously much more complex to interpret as a human being. In short, a model seeks a balance between correctly representing reality and being simple so that it can be used \cite{kuhne}.
\newline

Let's imagine that you want to model the weather. For this purpose, different pieces of evidence are collected, and, after observation, a first model can be made:

\begin{displayquote}
“The summer will be sunny, hot and clear. The winter will be cold, cloudy, and rainy.”
\end{displayquote}

If you keep collecting evidence, you will soon realise that this model is very simple as there will be days that are cold in summer and hot in winter. There will be storms in summer and clear days in winter. This will lead to improvements in the model in each iteration.
\newline


But if you keep studying other evidence in some tropics or in some poles, for example, you will conclude that the model was very simple and therefore more rules will have to be added to make this model more similar to reality anywhere in the world.
\newline

In the end, you will get a very complex model with all the exceptions and conditions. An alternative to this is to make use of probability to be able to say mathematically that most of the time a summer day will be sunny and not such a complex model to depend on.
\newline

Probability is the perfect tool to limit the uncertainty about a subject due to lack of knowledge or data or to avoid the work of making a complex model that would lead to less understanding for the human mind and dispersion in its approach. Being able to use a probability is much easier than having to study all the physical conditions of an environment and the behaviour of its entities in order to know for sure what is going to happen. Using probability to build models results in probabilistic models. These models compress many of the variability of our reality based on probabilities, making it easier to manage the information we receive from the environment \cite{kuhne}.
\newline

Our brain applies similar schemes to these probabilistic models and it is thanks to them that we have the ability to conceptualise, predict, generalise, reason or learn. For this reason, discovering what these models are is one of the basic objectives of the field of \acrlong{ml} and one of the fundamental tools of \acrshort{ai}.
\newline