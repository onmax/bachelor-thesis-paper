\subsection{Artificial Intelligence}

\acrfull{ai} belongs to the field of Computer Science that has achieved great recognition and popularity during the last decade. Despite being a discipline that has been developed over the last five years, it has already managed to provide solutions to a large number of problems that in the past were considered unsolvable or highly complex. Achievements such as transcription from voice to text \cite{voice2text} or vice versa \cite{text2voice}, automation of processes with robots \cite{robots}, automatic driving \cite{automaticdriving}, style transfer \cite{styletransfer} are some of the most famous examples. The problems that \acrshort{ai} attempts to solve cover problems that affect any area of life.
\newline

Providing an exact definition of \acrshort{ai} is difficult, since it is a concept that depends on the very definition of intelligence that today has multiple interpretations. Many authors propose their own definition of AI \cite{haugeland, bellman, charniak, winston, kurzweil, knight, nilsson} and if we take all of them, we could extract a common idea: AI is a discipline of computer science that aims to create entities that can \textbf{imitate} intelligent behaviour, from analysing patterns, classifying elements in an image, predicting values to recognising voices, driving, trading on the stock exchange and many other entities in which a system can simulate intelligent behaviour.
 \newline
 
The definition of \acrshort{ai} can mainly be grouped into one of the following approaches \cite{amodernapproach}:

\begin{enumerate}
\item Systems that think like humans \cite{haugeland, bellman}: The capacities of the system must be specific to human beings. Alan Turing (1950) proposed the Turing Test [14], which aims to test these capacities and if it is passed, it will be demonstrated that the system will behave like a human being. The reliability of this test is questionable since it is easy to create a system that imitates human behaviour. Imitating is not having intelligence of your own \cite{amodernapproach}.

\item Systems that act like humans \cite{kurzweil, knight}: Turing's test did not consider the physical person but only their intelligence. Hanard (1991) proposed Turing's total test \cite{harnad} which considers all kinds of external stimuli, perception skills and object manipulation. In other words, it adds new requirements to the system: computer vision and robotics that require new capabilities: robotics, natural language processing, knowledge representation, automated reasoning, and automatic learning. These six disciplines are six of the AI disciplines currently being studied as will be explained later \cite{amodernapproach}.

\item Systems that think rationally \cite{charniak, winston}: The system would try to imitate the structure and processing of information as a human would. This would require research and further development of precise theory of mind, in order to translate this theory into a system. This could be done either by an introspective study on an individual basis or through psychological experiments that serve to make different definitions and requirements so that the system can be programmed. To carry out a test, the same data would be provided with a structure and reaction times would be measured. If the structure and output data, together with the reaction time are similar to those of a human, there is evidence that some of the programmed mechanisms can be compared with those used by a human and therefore be considered intelligent \cite{amodernapproach}.

\item Systems that act rationally \cite{poole, nilsson}: It would no longer be systems that are built but agents. An agent is something that reasons (agent comes from the Latin \textit{agere}, to do). This agent is expected to act with the intention of achieving the best result, and can even take the decision to inaction, if it considers it appropriate in situations of uncertainty.
 
It is considered that this agent must be able to make inferences when establishing which actions will help to achieve the desired objective. To obtain a correct inference not only depends on rationality, since sometimes the best decision will not be the consequence of a slow process of deliberation but of a reflex action or even situations where there is nothing right to do and a decision has to be made. This approach is the most general and complex at a technical level because it proposes the development of different parts: a rational part that is common to the other approaches and a part of logical inference. On the other hand, it is simpler to carry out tests with these agents since rationality is well defined mathematically, and therefore it can be broken down to design agents that can achieve this in a verifiable way \cite{amodernapproach}.

\end{enumerate}

Today, created artificial intelligences are what is known as the weak type. These are systems that can exceed the capabilities of a human in some task. However, if you select one of these systems that excels in a very specific domain and try to get it to perform another task, the result of that task will be much worse than that of a human being. This ability to be able to multi-task is a much sought-after feature that is still being researched in all \acrshort{ai} departments today. At the moment, there are only weak \acrshort{ai}'s capable of doing one or more groups of tasks extraordinarily well. In contrast, systems with strong \acrshort{ai} are systems that can carry out tasks in different domains, but there is no example of such a system yet \cite{amodernapproach}.
\newline

It is important to mention that some of the four approaches to defining AI explained above are aimed at imitating intelligent behaviour. Trying to imitate a behaviour can be relatively easy. For example, a system can be programmed to play chess according to a set of rules and predicates. Being a machine, it may have memorised similar games in the past and know what strategies its opponent took and thus have access to much more information than its opponent. In this way, the system will be able to choose the best decision in each move and will be unbeatable, but if we try to make it play with a strategy that it has not seen before, it will not be able to use a strategy that it has tactically valid, that is to say, it will not work for other cases because the system is only imitating and not learning.
\newline

Imitating does not mean that such behaviour is in essence cognitive behaviour, however, according to the definition given of \acrshort{ai}, the system playing chess would be considered a system with \acrshort{ai}, even if the mechanics of the move have been memorised. That is why within this field of computer science we can find different subcategories that respond to different intelligent behaviours. These subcategories are:
\begin{itemize}
\item Robotics.
\item Computer vision.
\item \acrlong{nlp}.
\item Knowledge Representation.
\item Automatic Reasoning.
\item Automatic Learning.
\end{itemize}

Above all, if there is one category that defines us as intelligent agents, it is the ability to learn. This capacity is the focus of this work and is the object of study of \acrlong{ml} which we will see in more detail below.
\newline