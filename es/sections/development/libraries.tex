\subsection{Herramientas y librerías usadas} \label{used_libs}

El lenguaje usado para el desarrollo de esta librería ha sido Python 3.6 \cite{python}. Este lenguaje es una opción de \textit{facto} junto a $R$ en el campo \acrlong{ai}. Junto con Python, se han desarrollado un conjunto de librerías que favorecen que este lenguaje sea la opción favorita para desarrollar proyectos de este tipo. Las librerías usadas en especifico para es proyecto han sido:
 
\begin{itemize}
    \item Tensorflow \cite{tensorflow2015-whitepaper}: es la plataforma de \acrlong{dl} más importante del mundo. Esta librería \textit{open-source} de Google va más allá de la \acrlong{ai}, pero su flexibilidad y gran comunidad de desarrolladores lo ha posicionado como la herramienta líder en el sector del \acrlong{dl}.
    \item Keras \cite{keras}: Keras es una biblioteca de código abierto escrita en Python, que se basa principalmente en el trabajo de François Chollet, un desarrollador de Google, en el marco del proyecto ONEIROS. El objetivo de la biblioteca es acelerar la creación de redes neuronales: para ello, Keras no funciona como un framework independiente, sino como una interfaz de uso intuitivo (API) que permite acceder a varios frameworks de aprendizaje automático y desarrollarlos. Entre los frameworks compatibles con \textit{Keras}, se incluyen TensorFlow.
    \item Numpy \cite{numpy}: Numpy es la librería por excelencia para computación científica en Python. Trae integradas muchas funciones de cálculo matricial de N dimensiones, así como la transformada de Fourier, múltiples funciones de álgebra lineal y varias funciones de aleatoriedad.
    \item Pandas \cite{pandas}: Es una biblioteca de código abierto que proporciona estructuras de datos y herramientas de análisis de datos de alto rendimiento y fáciles de usar para el lenguaje de programación Python.
    \item Matplotlib \cite{matplotlib}: Es una biblioteca para la generación de gráficos a partir de datos contenidos en listas o arrays en el lenguaje de programación Python y su extensión matemática \textit{NumPy}. Proporciona una API, pylab, diseñada para recordar a la de MATLAB. Se ha usado el tema gráfico llamado SciencePlots \cite{SciencePlots}.
\end{itemize}