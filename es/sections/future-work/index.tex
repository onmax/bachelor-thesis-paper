\section{Trabajo futuro}\label{future_work}

Gran parte de este proyecto ha sido en el estudio teórico del concepto de las redes neuronales mientras que la aplicación práctica de lo aprendido ha sido la otra parte del tiempo dedicado. En esta parte práctica, principalmente, se ha invertido tiempo tanto en el aprendizaje de las librerías que se usan en la industria como pueden ser \textit{Tensorflow} y \textit{Pandas}, al igual que la dedicación en el entendimiento del problema y estudio de diferentes modelos y estudiar su comportamiento. Con este trabajo por lo tanto ha quedado demostrado que el mejor modelo que ha funcionado ha sido el auto-regresivo.
\newline

Como parte del trabajo futuro se propone mejorar este modelo para minimizar todo lo posible las métricas y proponer un modelo que obtenga resultados parecidos o mejores al estado del arte actual. Para ello, se podría estudiar más en profundidad tanto los parámetros de la red como la tasa de aprendizaje, el optimizador o el error. Pero, añadiendo más información al \textit{dataset} podría ser más útil, puesto que a mayor cantidad de datos, más preciso será el modelo. Se podrían usar por ejemplo variables como si es día festivo o la condiciones meteorológicas. Y esta es una de las principales diferencias que existen entre el \textit{dataset} usado en este proyecto y en otros. El \textit{dataset} usado solo contaba con tres variables sobre la información temporal: \textit{hour}, \textit{day\_of\_month} y \textit{month} y por lo tanto, no cabe duda que añadiendo estas cantidades de variables que aporten información sobre los patrones de alquiler de bicicletas, el modelo mejorase considerablemente.
\newline

Otro aspecto a mejorar de este proyecto sería la explotación de los resultados en algún tipo de herramienta útil a la empresa o a los usuarios. Los modelos entrenados simplemente han sido usados para computar las métricas y posteriormente han sido descartados puesto que no había planes de usarlos en producción. Por el contrario, si hubiese existido algún proyecto haciendo uso de los modelos, estos se podría haber guardado en un fichero y se podría usar en diferentes herramientas. Por ejemplo, una aplicación web que indique al usuario la predicción de bicicletas disponibles para cierta hora o un software que permite a los gestores de logística reubicar las bicicletas de forma más eficiente. Estas y otras aplicaciones podrían usar como módulo principal algún modelo de red neuronal explicado en este proyecto.
\newline



to\_station\_id flujo en la red


Por otro lado, se podrían usar más variables conteniendo información importante como puede ser el calendario laboral o variables relacionadas con la meteorología, las cuales influyen en los patrones de comportamiento de la red de bicicletas.
\newline

Por ejemplo, un lunes no laboral, la cantidad de bicicletas alquiladas en un parque será mayor que cualquier otro lunes de otra semana. U otro ejemplo, si un día las condiciones meteorológicas nos son favorables para el uso de bicicleta, la cantidad de bicicletas alquiladas se reducirán considerablemente. Este tipo de variables son algún ejemplo que se podría añadir al \textit{dataset} y así poder mejorar la precisión del modelo pero por falta de tiempo y por simplicidad no se han añadido. Además, los resultados obtenidos de por sí se han considerado bastante buenos y por lo tanto, no se ha visto la necesidad de invertir tiempo en este apartado aunque sería un buen estudio como mejoraría estos modelos con dichos cambios.
\newline


Finalmente, sería interesante estudiar si es factible poder usar este proyecto pero usando otros datasets y probando así su portabilidad.


% TODO EXLICAR QUE OPTIMIZADOR HA SIDO USADO

