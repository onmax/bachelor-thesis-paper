\subsection{Redes neuronales artificiales. Entrenamiento}

Las redes neuronales son algoritmos que se usan en el campo de la \acrshort{ai} y tratan solventar problemas en los cuales se trabaja con gran cantidad de datos tratando de buscar patrones en ellos. Además, es una de las pocas alternativas a otros algoritmos que nos son capaces de tratar gran volumen de datos. 
\newline

Los inicios de las redes neuronales se remontan a la década de 1950, cuando McCulloch y WalterPitts\cite{kleene} trabajaron en un modelo matemático que se asemejaba al comportamiento que conocían de una neurona. El científico Frank Rosenblatt, inspirado en este trabajo, desarrollo lo que se conoce como redes de perceptrones. Este fue el primer acercamiento a lo que hoy se conoce como redes neuronales\cite{nielsen}. 

\subsubsection{Entrenamiento de una red}\label{training}
Como resumen de visto hasta ahora, para crear una red es necesario tener cuatro elementos distintos:
\begin{itemize}
\item Valores de entrada: Información con la que se va a predecir un dato o un conjunto de datos.
\item Estructura de la red: Esta es una información que se debe de conocer a priori antes de crear el modelo. Tiene distintas propiedades:
\begin{itemize}
    \item Número de capas: A mayor número de capas, mayor será el tiempo que tardará el modelo en computar el vector de salida $y$ porque mayor cantidad de cálculos tendrá que realizar. El número de capas junto con el número de neuronas por capa son parámetros importantes, puesto que un número muy bajo en el modelo y este no tendrá una buena precisión. Por lo contrario un número muy alto puede producir lo que se conoce como \textit{overfitting}(ver sección \ref{overfitting}).
    \item Número de neuronas en cada capa: Cabe destacar que el número de neuronas en la última capa será el tamaño del vector $y$, es decir, el número de etiquetas que el modelo predecirá.
    \item Función de activación por cada capa.
    \item Arquitectura de la red. % Se puede ver más información sobre los tipos de arquitecturas de redes en la sección TODO.
\end{itemize}
\item Las matrices $W$: Son matrices que recogen la información asociada a cada capa sobre los pesos $w$ y \textit{bias} $b$ de cada neurona de la red.
\end{itemize}

Las matrices $W$ son matrices con valores creados aleatoriamente. El proceso de entrenamiento de la red tratará de optimizar estas matrices para que el vector $y$ resultante sea lo más preciso posible. Para poder entrenar un modelo es necesario tener previamente un \textit{dataset} con un conjunto de vectores $x$ y su valor real. La cantidad de datos que proporcionemos al modelo para que aprenda está relacionada con la precisión del modelo. Con el ejemplo descrito en la Tabla \ref{tab:houses} se podrían usar las columnas \textit{tipo}, \textit{coords}, \textit{gc}, \textit{m2} y \textit{h} para predecir el valor $precio$.
\newline

Básicamente la red inicializa la matriz $W$ de forma aleatoria. El modelo dado un vector $x$ realiza todos los cálculos de cada neurona y devuelve un valor $y$. $y$ es lo que el modelo ha predicho. Este proceso es el que se conoce como \textit{forward-pass}.
\newline

Para que la red aprenda, necesita primero saber si se ha equivocado y la magnitud del error. Si la magnitud de este error es muy grande, se deberá ajustar los valores de $W$ para minimizar el error. Por lo contrario, si el error es pequeño, no se ajustarán mucho los valores de $W$ porque realizando un ajuste en $W$ puede provocar una ligera mejoría en dicha predicción, pero puede desajustar otras predicciones que el modelo ha hecho previamente y eran también bastantes precisas. La elegancia de este algoritmo es que encuentra un balance entre lo que es ajustar valores para que la red aprenda y al mismo tiempo no desajustar demasiado para que la red se descompense en su aprendizaje global. Es la aplicación de la imitación del aprendizaje humano, un error con una gran repercusión deberá modificar comportamientos futuros, un error sin repercusión podrá ser ignorado o incorporar cambios proporcionados a las repercusiones.
\newline

El error se cuantifica con una función llamada función de coste o función de pérdida explicada en la sección \ref{costfunction}. A partir del valor del error, se usará un algoritmo que tratará de calcular la responsabilidad de cada neurona en dicho error y de esta forma poder ajustar los pesos y la \textit{bias} asociadas a dicha neurona. Este algoritmo se llama \textit{backpropagation} y es explicado en la sección \ref{backpropagation}.
\newline

Se puede pensar que este proceso se puede repetir infinitamente hasta tener un modelo perfecto, pero como se ve en la sección de \ref{overfitting} esto puede provocar que la red memorice el dataset que es usado para entrenar y no tenga la capacidad de generalizar. Por lo tanto, no solo se trata de ejecutar el algoritmo, sino que hay que realizar ciertas optimizaciones y tener varios conceptos en cuenta como por ejemplo: la selección de función de activación, diseño del vector de entrada y salida, métricas a usar, entre otros.
\newline

\label{p:company_backpropagation}
Como analogía para entender el algoritmo de \textit{backpropragation}, se puede usar la jerarquía de una empresa. Dicha empresa tras un trimestre desastroso elabora un resumen con los resultados. Estos resultados son equivalentes al error que el modelo produce. El jefe (la última capa de la red), tratará de rendir cuentas con los directivos. Estos directivos a su vez tratarán con otros directivos con menor responsabilidad y estos a su vez lo harán con jefes que estén por debajo suya y así sucesivamente hasta llegar al último nivel en la empresa (propagando el error hacia las capas más básicas). Posteriormente, la empresa elaborará un informe estudiando la responsabilidad de cada persona en el resultado del trimestre (que en el algoritmo de \textit{backpropagation} es conocido como el vector gradiente). Ese informe llegará al departamento de recursos humanos y este, tratará de modificar el comportamiento de cada trabajador en la empresa en función de su responsabilidad en el error (algoritmo del descenso del gradiente).
\newline

En el siguiente diagrama se puede visualizar el proceso que se lleva a cabo para entrenar una red neuronal: 
\begin{figure}[H]
    \centering
    \includegraphics[width=15cm]{images/state-of-art/training/training.png}
    \caption{Proceso de entrenamiento de una red neuronal}
    \label{fig:error_regression}
\end{figure}

El entrenamiento de una red es un proceso iterativo. En cada iteración, o también conocido como \textit{epoch}, se tomarán un conjunto de datos del \textit{dataset} de entrenamiento de forma aleatoria. Cada uno de estos conjuntos de datos tomados para cada epoch se le denomina \textit{batch} y el tamaño del \textit{batch} suele ser un parámetro puesto por el usuario. El tamaño de un \textit{batch} suele ser de 32, 64 o 128 valores.
\newline

En cada \textit{epoch}, el modelo solo trabajará con dichos datos. Predecirá un valor para cada una de las entradas y junto al valor real, la función de coste, el algoritmo de \textit{backpropagation} y el descenso del gradiente, se irán ajustando de forma iterativa las matrices $W$.

\subsubsection{Cost function}\label{costfunction}

For the model to learn, it is first necessary to be able to identify errors in the process. The cost function is a function that will calculate the error that is occurring in a model. This function will have two parameters: The expected value and the value calculated by the model. The difference between both values is what is known as error or loss, so this function is also known as loss function or objective function.
\newline

The simplest error is the error given by the difference between the expected value and the actual value:
\newline
\begin{equation}
    \begin{split}
    c_i & = Real_{value} - Expected_{valuue} = \hat{y_i} - y_i, \\ 
    \text{where}~c_i &= \text{The value of the sample loss} \\
    i &= \text{i\textsuperscript{th} sample of the dataset} \\
    ~\hat{y_i} &= \text{Result of the model} \\
    ~y_i &= \text{Real value}
  \end{split}
\end{equation}

Using the example seen before in the regression of figure \ref{fig:regression} the error for the data $i$ seen graphically would be as follows:

\begin{figure}[H]
    \centering
    \includegraphics[width=10cm]{images/state-of-art/cost-function/error_function.png}
    \caption{Regression error for data $i$}
    \label{fig:error_regression}
\end{figure}

In a network you need to know the error of the layer and not the error of each neuron. To do this some function will be applied that receives a vector as input (the errors of each neuron in a layer) and a single output value (layer error). Some of the most commonly used functions are listed below \cite{tensorflow2015-whitepaper}:


\begin{itemize}
\item \acrfull{mae} \cite{errors_basics} \label{MAE_loss}: This is the simplest regression error metric to understand. The absolute error of each data is taken, so that negative and positive errors are not cancelled out. Then the arithmetic mean is calculated. In fact, the \acrshort{mae} describes the typical magnitude of the residuals. The equation is as follows:

\begin{equation}
\centering
    \begin{split}
        \text{\acrshort{mae}} = c_i(y_1, \hat{y_1}) = \frac{1}{n} \sum_{i=1}^n |y_i - \hat{y_i}|
    \end{split}
\end{equation}

\begin{figure}[H]
    \centering
    \includegraphics[width=7cm]{images/state-of-art/cost-function/mae.png}
    \caption{Visualisation of the \acrshort{mae}. The error is the arithmetic mean of all errors.}
    \label{fig:error_mae}
\end{figure}

\item \acrfull{mse} \cite{errors_basics}\label{MSE_loss}: This equation calculates the average of all squared errors. By squaring it is possible to penalise with greater intensity those points that are further away from the estimation of the linear regression and with less intensity those that are closer. This equation is widely used when the task to be solved is of a regression type. The equation is as follows:
\begin{equation}
\centering
    \begin{split}
        \text{\acrshort{mse}} = c_i(y_1, \hat{y_1}) &= \frac{1}{n} \sum_{i=1}^n (\sqrt{(\hat{y_i} - y_i)^2}^2) \\
        & = \frac{1}{n} \sum_{i=1}^n (\hat{y_i} - y_i)^2
    \end{split}
\end{equation}

\begin{figure}[H]
    \centering
    \includegraphics[width=7cm]{images/state-of-art/cost-function/mse.png}
    \caption{Visualisation of the \acrshort{mse}. The error is the average of the areas of the squares.}
    \label{fig:error_mae}
\end{figure}

\item \acrfull{rmse} \cite{errors_basics} \label{RMSE_loss}: This equation is the square root of \acrshort{mse}. If compared at the value level, they are interchangeable although they use different scales. The equation is as follows:
\begin{equation}
\centering
    \begin{split}
        \text{\acrshort{rmse}} = \sqrt{\text{\acrshort{mse}}} = \sqrt{\frac{1}{n} \sum_{i=1}^n (\hat{y_i} - y_i)^2}
    \end{split}
\end{equation}

\item \acrfull{msle} \cite{errors_basics} \label{MSLE_loss}: This equation is similar to \acrshort{mse}. It is often used when it is known in advance that the results are normally distributed and large errors are not intended to be significantly more penalised than small ones. The equation is as follows:
\begin{equation}
\centering
    \begin{split}
        \text{\acrshort{msle}} = c_i(y_1, \hat{y_1}) &= \frac{1}{n} \sum_{i=1}^n (\log{y_i + 1} - \log{\hat{y_i} + 1}) \\
        & = \frac{1}{n} \sum_{i=1}^n \left(\log{\left(\frac{y_i + 1}{\hat{y_i} + 1}\right)}\right)^2
    \end{split}
\end{equation}


\item Huber loss \cite{huber_loss}\label{huber_loss}: It is already known that \acrshort{mse} is better for learning outliers in the data set, on the other hand, \acrshort{mae} is good for ignoring outliers. But in some cases, data that appear to be outliers do not bother and should not be given high priority. The loss of Huber is a combination of \acrshort{mse} and \acrshort{mae}. $\delta$ will be used to define a bias for using \acrshort{mse} or \acrshort{mae}. The equation is as follows:

\begin{equation}
\centering
    \begin{split}
    \text{Huber\_loss} = c_i(y_1, \hat{y_1}) &= \left\{ 
        \begin{array}{cl} 
            \text{\acrshort{mse}} & \text{if }|\hat{y}-y| \le \delta, \\
            \text{\acrshort{mae}} & \text{e.o.c.}
        \end{array}\right. \\ &= \left\{ 
        \begin{array}{cl} 
            \frac{1}{n} \sum_{i=1}^n (\hat{y_i} - y_i)^2 & \text{if }|\hat{y}-y| \le \delta, \\
            \delta \left(\frac{1}{n} \sum_{i=1}^n |y_i - \hat{y_i}|-\frac{\delta}{n}\right) & \text{e.o.c.}
        \end{array}\right.
    \end{split}
\end{equation}

\begin{figure}[H]
    \centering
    \includegraphics[width=7cm]{images/state-of-art/cost-function/huber.png}
    \caption{Visualisation of Huber loss \acrshort{mse} is calculated if $|\hat{y}-y| \le \delta$, in other
case is calculated \acrshort{mse}.}
\label{fig:error_mae}
\end{figure}


\item \acrfull{mape} \cite{errors_basics} \label{MAPE_loss}: This is the equivalent percentage of \acrshort{mae}. The equation is equal to the \acrshort{mae}, but with adjustments to convert the values into percentages. They allow to see the distance between the model results and the real result showing the data in an easier way to interpret for human beings. The equation is as follows:

\begin{equation}
\centering
    \begin{split}
        \text{\acrshort{mape}} = c_i(y_1, \hat{y_1}) &= \frac{100\%}{n} \sum_{i=1}^n \left|\frac{y_i-\hat{y_i}}{y}\right|
    \end{split}
\end{equation}

\begin{figure}[H]
    \centering
    \includegraphics[width=7cm]{images/state-of-art/cost-function/mape.png}
    \caption{\acrshort{mape} visualisation. The error is the mean of the error proportions to with respect to the value $y$.}
    \label{fig:error_mae}
\end{figure}

\item \acrfull{mpe} \cite{errors_basics} \label{MPE_loss}: This is exactly what \acrshort{mape} is, but without the absolute value. Since positive and negative errors are cancelled out, no statement can be made about the overall performance of the model's predictions. However, if there are more negative or positive errors, this bias will be shown in the \acrshort{mpe}. Unlike \acrshort{mape} and \acrshort{mae}, the \acrshort{mpe} is useful because it allows one to see whether the model systematically underestimates (more negative errors) or overestimates (positive errors). The equation is as follows:

\begin{equation}
\centering
    \begin{split}
        \text{\acrshort{mpe}} = c_i(y_1, \hat{y_1}) &= \frac{100\%}{n} \sum_{i=1}^n \frac{y_i-\hat{y_i}}{y}
    \end{split}
\end{equation}

\begin{figure}[H]
    \centering
    \includegraphics[width=7cm]{images/state-of-art/cost-function/mpe.png}
    \caption{Visualisation of the \acrshort{mpe}. It shows the number of errors that are positive and the number that are negative.}
    \label{fig:error_mae}
\end{figure}


\item Loss of cross entropy: Used explicitly to compare a probability of "fundamental truth" ($y$ or "targets") and some predicted distribution ($\hat{y}$ or "predictions"). It aims to calculate the average of the probabilities that a value belongs to one class or another, so it is very useful for classification problems. It is widely used when the softmax activation function is used, since it also works with probabilities.
\begin{equation}
\centering
    \begin{split}
        c_i(y_1, \hat{y_1}) &= - \sum_i y_{i,j}log(\hat{y_{i,j}})\\
        \text{where}~j &= \text{"True" probability index}
    \end{split}
\end{equation}


The "true" probability is a vector with all values at $0$ except one that has the value equal to $1$. This type of vector is known as an one-hot vector, where a value is hot if it is equal to $1$ or cold if it is equal to $0$. When the results of the model are compared with a one-hot vector using the cross entropy, the values equal to $0$ are not used, and the log loss of the target probability is multiplied by $1$, making the calculation of cross entropy relatively simple. This is also a special case of the calculation of cross entropy, called category cross entropy.
\newline

An example of a one-hot vector would be the following: $[0,1,0,0]$ where for example the following classes would be represented: Dog, cat, horse and elephant. The $1$ represents that the data given to the model represents a cat. These types of vectors are widely used in classification tasks and in \acrfull{nlp}.
\newline

There is a subtype called binary cross entropy loss. This type of error can be calculated when trying to classify only with two types: $0$ or $1$. It is usually used as if it were a boolean value in a programming language. A \small{\verb|true|} \normalsize if it is type A or a \small{\verb|false|} \normalsize if it is not type A. For example: cat or no cat or indoor or outdoor. The mathematical equation is as follows:

\begin{equation}
\centering
    \begin{split}
        c_{i,j}(y_1, \hat{y_1}) &= (y_{i,j})(-log(\hat{y}_{i,j})) + (1-y_{i,j}) (-log(1 - \hat{y}_{i,j})) \\
        &= -y_{i,j} \cdot log(\hat{y}_{i,j}) - (1 - y_{i,j}) \cdot log(1 - \hat{y}_{i,j})
    \end{split}
\end{equation}


\end{itemize}


\subsubsection{Minimizando el error}\label{minimizing-error}

Una vez elegida alguna de las funciones de coste, el objetivo es minimizar el error y así mejorar el modelo:

\begin{equation}
\begin{split}
        W^* &= \argmin_{W} \frac{1}{n} \sum_{i=1}^n c(a(x^{(i)}; W), y^{(i)}) \\
        &= \argmin_{W} c_i(W) \\
        \text{donde}~x^{(i)} &= \text{Vector de entrada para el dato }i \\
        y^{(i)} &= \text{Vector real para el dato }i \\
        W &= \text{Matrices con los pesos y bias de cada cada} 
  \end{split}
\end{equation}

Para calcular el mínimo de una función, se necesita calcular la derivada de dicha función e igualarla a $0$. Minimizando la función de coste, es decir, derivando la función de coste, también se minimiza el error. Para ello hay distintas formas de minimizar el error:
\begin{itemize}
\item Minimizar función de coste

Este fue el algoritmo usado en las redes de perceptrones para calcular la matriz $W$. El funcionamiento de este algoritmo es sencillo. Partiendo de la definición de una función de coste cualesquiera que permita cuantificar el error del modelo, se tratará de minimizar dicho valor. Una de las ecuaciones más populares es la que se conoce como el \acrlong{ols}. Esta función parte de la derivada de \acrshort{mse}:

\begin{equation}
    \begin{split}
    L_i & =  (\hat{y_i} - y_i)^2 \\
     & = (y - x \cdot W)' \cdot (y - x \cdot W) \\
     & = y'y - W'x'y - y'xW + W'x'x W
  \end{split}
\end{equation}

Derivando y despejando:
\begin{equation}
    \begin{split}
L_i'() = & -2x'y + 2x'xW \\
W^* = & (x'x)^{-1}x'y
\end{split}
\end{equation}

Esta ecuación permite el cálculo de la matriz $W$ óptima posible solo usando los valores de entrada y de salida de modelo como argumentos. Aunque este algoritmo de entrenamiento tiene varias limitaciones:
\begin{itemize}
\item No es extensible a redes más complejas.
\item El cálculo de la matriz inversa es muy costoso computacionalmente.
\item Se ha usado el error mínimo cuadrado, una de las más simple, puesto que es una función con forma convexa y por tanto una derivada fácil de calcular, pero hay otras funciones de coste que no permiten minimizar el error con esta técnica.
\newline

Estas limitaciones, demostradas matemáticamente en el libro "\textit{Perceptron}"\cite{papert} de Minsky y Papert (1969), provocó un corte repentino en la financiación de proyectos de inteligencia artificial y más específicamente en aquellos relacionados con los sistemas de redes neuronales durante un periodo de más de 15 años conocidos como el invierno de la inteligencia artificial.
\end{itemize}


\item Descenso del gradiente

Este algoritmo no es una fórmula como los mínimos cuadrados ordinarios, sino un método iterativo que poco a poco va minimizando el error. De algún modo se puede asimilar a como aprendemos los seres humanos: No con una única fórmula, sino a través de la experiencia reduciendo nuestros errores con el tiempo.
\newline

El cálculo del gradiente es un proceso que se puede representar de la siguiente forma:
\begin{figure}[H]
    \centering
    \includegraphics[width=13cm]{images/state-of-art/gradient-descent/gradient-algorithm.png}
    \caption{Proceso iterativo para minimizar el error}
    \label{fig:gradient_descent}
\end{figure}

El método de los \acrshort{ols} minimiza el error igualando la derivada a $0$ para poder hallar los mínimos de \acrshort{mse}. Con esto, se pueden calcular los mínimos locales y globales, pero también se puede calcular máximo locales, puntos de inflexión o puntos de silla provocando un sistema de ecuaciones grande y bastante ineficiente de resolver \cite{papert}.
\newline

Una forma de entender este método es el siguiente: Una persona se encuentra en un terreno montañoso y el objetivo de esta persona es llegar al punto más bajo. Para ello, analizará el terreno donde se encuentra y evaluará la pendiente y se moverá hacia donde la pendiente desciende con mayor intensidad. Descenderá una cantidad de pasos y repetirá el proceso: analizar la inclinación y descender. Esto se repetirá hasta que llegué a lo más abajo posible y no haya forma alguna de seguir bajando. Esta es la lógica del algoritmo del descenso del gradiente.

\begin{figure}[H]
    \centering
    \includegraphics[width=7cm]{images/state-of-art/gradient-descent/gradient.png}
    \caption{Ejemplo gráfico de la iteración del descenso del gradiente}
    \label{fig:gradient_descent}
\end{figure}


El terreno en esta metáfora sería la función de coste. La persona es el valor calculado por la función de coste, el cual se quiere minimizar y por ello se evalúa la pendiente o gradiente solo en ese punto y de ese modo no se depende de la derivada de la función de coste sino de la derivada parcial respecto a cada uno de los valores de entrada en la neurona. El número de pasos que la persona bajará es un valor conocido como tasa de aprendizaje (\textit{learning rate} en inglés) y es un parámetro más de la red neuronal. 
\newline


En la figura \ref{fig:gradient_descent}, hay dos ejes, representando una neurona que tiene solo un argumento de entrada. El eje de ordenadas representa el error de la neurona. Se ha representado un espacio bidimensional, pero se puede usar cualquier dimensión puesto que el número de entradas de la neurona no está limitado. Usando un espacio tridimensional, la función de coste sería una superficie irregular. De hecho, el vector $\nabla f$ siempre tendrá como mínimo dos valores, un peso $w$ y la bias $b$.
\newline

En este trabajo no se explica cómo calcular la derivada de una función o derivada parcial($\sigma$). Dicho esto, matemáticamente, el proceso de este algoritmo es el siguiente \cite{}:
\begin{enumerate}
\item Se inicializa los pesos $w$ y bias $b$ de la neurona quieren ajustar de forma aleatoria:
\begin{equation}
    ~N(0, \sigma_2)
\end{equation}

\item Se realiza un bucle hasta la convergencia:
\begin{enumerate}

\item Se calculan derivadas parciales para cada uno de los parámetros. Una derivada parcial mide cuanto impacto tiene una única variable de entrada en la salida de la función y se calcula igual que una derivada, lo único que se repite la derivada para cada variable de entrada. Cada uno de esos valores indicará cual es la pendiente en el eje de dicho parámetro relacionado a un único valor de entrada.

\begin{equation}
    \begin{split}
    \nabla f &= \nabla f_b \oplus \nabla f_w \\\
    \nabla f_b &= \begin{pmatrix} \frac{\partial c}{\partial b} \end{pmatrix}\\
    \nabla f_w &= \begin{pmatrix} \frac{\partial c}{\partial w_1}, & \frac{\partial c}{\partial w_2}, & \cdots , &  \frac{\partial c}{\partial w_n} \end{pmatrix}
  \end{split}
  \label{eqn:gradients}
\end{equation}


Conjuntamente todas las direcciones, es decir, todas las derivadas parciales conforman un vector que indica la dirección a la que la pendiente asciende, este vector es también conocido como gradiente($\nabla f$) y el cual tendrá el mismo número de elementos que el vector de entrada y cada valor contendrá la solución a la derivada parcial con respecto a cada uno de los valores de entrada.
\newline

El objetivo de esta función es minimizar y no maximizar, por lo que se negará el gradiente para indicar la dirección en la que la pendiente desciende.

\begin{equation}
    \nabla f = -\nabla f_b \oplus \nabla f_w
\end{equation}

\item Se actualizan los pesos con los nuevos valores del gradiente negativo multiplicado por la tasa de aprendizaje. La tasa de aprendizaje es simplemente un valor que determina como de grande será el paso en cada iteración que se verá con mayor profundidad en la sección \ref{learningrate}:
\begin{equation}
    \begin{split}
    b^* = b - \eta * \nabla f_b \\
    w^* = w - \eta * \nabla f_w
    \end{split}
\end{equation}

Al igual que ocurría con el algoritmo de \textit{feed-forward}, el descenso del gradiente trabajará por capas, por lo que este último paso se debería representar de la siguiente forma:

\begin{equation}
    W^* = W - \eta * \nabla f
\end{equation}

Esta ecuación representa que se actualizarán los pesos de todas las neuronas en una capa.

\end{enumerate}
\end{enumerate}

Este método se aplicará a todas las neuronas de la red. Dado un error, el descenso del gradiente obtendrá un vector que contendrá la derivada de los parámetros respecto al coste. Es decir, el gradiente será un vector con distintos valores, cuanto más grande sea dicho valor, más corrección se debe aplicar al parámetro que corresponde. Visto gráficamente se puede ver de la siguiente forma:

\begin{figure}[H]
    \centering
    \includegraphics[width=7cm]{images/state-of-art/gradient-descent/gradient_diagram.png}
    \caption{Representación del descenso del gradiente en una neurona. Lo rojo es la representación del error.}
    \label{fig:gradient_descent}
\end{figure}

Las derivadas parciales que se deben resolver para obtener el vector gradiente $\nabla f$, indica como varía el coste ante el cambio de los parámetro $w$ y $b$. A continuación, se muestra una imagen de las distintas partes de las derivadas parciales:

\begin{figure}[H]
    \centering
    \includegraphics[width=13cm]{images/state-of-art/gradient-descent/dx.png}
    \caption{Derivadas parciales usadas en el descenso del gradiente.}
    \label{fig:gradient_descent}
\end{figure}

\end{itemize}



\subsubsection{Backpropagation}\label{backpropagation}

In 1986, Rumelhart and other researchers published "Learning representation by back-propagating errors" \cite{rumelhart}, a work that brought back the popularity of neural networks. The paper, based on other works based on automatic differentiation, showed experimentally how using a new learning algorithm could make a neural network self-adjust its parameters in order to learn an internal representation of the information it was processing, an algorithm known as backpropragation.
\newline

This algorithm is adaptable to any model and with it, ended what is historically known as the \acrshort{ai} Winter by obtaining new funding and new projects in the field of \acrfull{dl}.
\newline

The prediction of a model is obtained using an algorithm called forward-pass as explained in section \ref{feedforward}. This value should be adjusted as closely as possible to the real value. To do this, the $W$ matrices must be adjusted in an iterative method in a process known as training (see section \ref{training}). In each iteration of the training, the $W$ matrices will be adjusted, first using the cost function, which will show the magnitude of the error (see section \ref{costfunction}), then this error will be propagated backwards in the model to know the responsibility of each neuron using an algorithm known as backpropagation and finally, for each neuron the gradient vector $\nabla f$ will be calculated, which represents how the neuron has affected the final result and with this its different weights $w$ and bias $b$ will be adjusted (see section \ref{minimizing-error}).
\newline


The backpropagation algorithm can be represented graphically:

\begin{figure}[H]
    \centering
    \includegraphics[width=14cm]{images/state-of-art/back-propagation/network_descent_gradient.png}
    \caption{Representation of the backpropagation algorithm.}
    \label{fig:basic_network}
\end{figure}

In this network, the error of each neuron is represented by red circles. The error calculated by the cost function is the largest error which indeed is the error of the output of the model. From that point, it will be backward propagated and computed which part of the error belongs to each neuron up to the first layer. For example, in the $L_2$ layer, the main neuron that has mismatched parameters is $L_{2,1}$. Likewise, this error is mainly due to the $L_{1,1}$ neuron. Therefore, it is logical to think that we should adjust neurons such as $L_{2,1}$ or $L_{1,1}$, however, leave neurons such as $L_{1,2}$ or $L_{2,2}$ as they are.
\newline
 
 The simplest neural network that can be constructed with a single hidden neuron and an output neuron is defined as follows:

\begin{figure}[H]
    \centering
    \includegraphics[width=12cm]{images/state-of-art/back-propagation/basic_network.png}
    \caption{Basic neural network.}
    \label{fig:basic_network}
\end{figure}

To perform the backpropagation algorithm, first the loss must be computed with the $c_i$ function. Then, the backpropagation algorithm will "distribute" the loss of the model to the previous layers (always starting with the last layer). This computation expresses how important the $W$ matrices are by indicating how much impact they have on the model. For example, if you change any value of $W_2$ in any way, you will be able to know how it affects the outcome of the model's prediction. Mathematically:

\begin{equation}
\begin{split}
    \frac{\partial c}{\partial w^{L_2}} &= \frac{\partial c(a(z^{L_2}))}{\partial w^{L_2}} \\
    \frac{\partial c}{\partial b^{L_2}} &= \frac{\partial c(a(z^{L_2}))}{\partial b^{L_2}} 
    \label{eqn:backpropagationsimplenetworklayer2}
\end{split}
\end{equation}

The numerator is a composition of functions and is basically the calculation of the forward-pass algorithm (see section \ref{feedforward}). To calculate the partial derivative of a composition of functions, it is necessary to use a calculation tool known as the chain rule.
\newline


An example to understand the chain rule would be the following. The Sun is known to be $100$ times larger than the Earth. The Earth is $4$ times larger than the Moon. Knowing these two relationships, we can tell how much the Sun is bigger than the Moon. To do this we simply multiply $100 \times 4 = 400$. The Sun is therefore $400$ times bigger than the Moon. Defining this problem with equations:
\begin{equation}
\begin{split}
    \frac{\partial T_{\text{Sun}}}{\partial T_{\text{Earth}}} = 100
    \qquad
    \frac{\partial T_{\text{Earth}}}{\partial T_{\text{Moon}}} = 4 \\
    \frac{\partial T_{\text{Sun}}}{\partial T_{\text{Moon}}} = \frac{\partial T_{\text{Sun}}}{\partial T_{\text{Earth}}} \cdot \frac{\partial T_{\text{Earth}}}{\partial T_{\text{Moon}}} = 400 \nonumber
\end{split}
\end{equation}


It is called the chain rule because it is chained in relation to the common data, in this case $T_\text{Earth}$ is the common part that on one side is in the denominator and on the other in the numerator. Therefore, to derive the composition of functions seen in equation \ref{eqn:backpropagationsimplenetworklayer2}, we simply have to multiply each of the intermediate derivatives. Recalling the equations of the forward-pass algorithm and the composition of functions to calculate the model error for the network shown in figure \ref{fig:basic_network}:


\begin{equation}
\begin{split}
    z^{L_2} = W^{L_2} \cdot y^{L_1} + b^{L_2}
    \qquad
    c(a^{L_2}(z^{L_2}))
\end{split}
\end{equation}

It is calculated how the error is propagated from the $L_2$ layer to the $L_1$ layer:

\begin{equation}
\begin{split}
     \frac{\partial c}{\partial w^{L_2}} &= \frac{\partial c}{\partial a^{L_2}} \cdot \frac{\partial a^{L_2}}{\partial z^{L_2}} \cdot \frac{\partial z^{L_2}}{\partial w^{L_2}} \\
     \frac{\partial c}{\partial b^{L_2}} &= \frac{\partial c}{\partial a^{L_2}} \cdot \frac{\partial a^{L_2}}{\partial z^{L_2}} \cdot \frac{\partial z^{L_2}}{\partial b^{L_2}}
\end{split}
 \label{eqn:backpropagationlayer2}
\end{equation}

These derivatives are easy to calculate and the explanation of each derivative would be as follows:


\begin{itemize}
\item First derivative. Derived from the activation function with respect to cost:

\begin{equation}
    \frac{\partial c}{\partial a^{L_2}}
\end{equation}

How does the cost of the network (it is the last layer) vary when the output of the network activation function is varied? In other words, the derivative of the cost function is calculated with respect to the output of the neural network. It is therefore necessary to know the derivative of the cost function used.

\item Second derivative. Derived from activation with respect to $z$:
\begin{equation}
    \frac{\partial a^{L_2}}{\partial z^{L_2}}
\end{equation}


It tries to reflect how the output of the neuron varies when the weighted sum of the neuron is varied. The derivative to be used is the derivative of the activation function. Derived from the sum of the weighted values of the weights and biases.

\item Third derivative: Derived from $z$ with respect to the parameters:
\begin{align*}
    \frac{\partial z^{L_2}}{\partial w^{L_2}} && \frac{\partial z^{L_2}}{\partial b^{L_2}}
\end{align*}

It indicates how the weighted sum $z$ varies with respect to a variation in the parameters. Parameter $b$ is an independent value so its derivative is a constant equal to 1. On the other hand, the derivative $\frac{\partial z^{L_2}}{\partial w^{L_2}}$ depends on the previous layer.


\begin{align*}
    \frac{\partial z^{L_2}}{\partial w^{L_2}} = a^{L_1} && \frac{\partial z^{L_2}}{\partial b^{L_2}} = 1
\end{align*}

\end{itemize}

Based on this definition, the first and second derivatives can be combined into one:

\begin{equation}
     \frac{\partial c}{\partial a^{L_2}} \cdot \frac{\partial a^{L_2}}{\partial z^{L_2}} = \frac{\partial c}{\partial z^{L_2}}
\end{equation}

This derivative indicates the degree to which the error is modified when there is a change in the error when there is a small change in the sum of the neurons in the layer. If the value is large it means that a small change in any of the neurons in the layers in the $z$ value will be reflected in the result of the model. On the contrary, if the value is small, a large change in any of the neurons in the $z$ value of will not affect the final result. In summary, this derivative shows how the layer (or some neuron in particular since it is a vector) affects the final result and therefore the error of the network and in that way the algorithm of the descent of the gradient will modify the parameters to be able to obtain the best possible result. This derivative is also known as the layer error and is represented by $\delta$:

\begin{equation}
    \frac{\partial c}{\partial a^{L_2}} \cdot \frac{\partial a^{L_2}}{\partial z^{L_2}} = \frac{\partial c}{\partial z^{L_2}} = \delta^{L_2}
\end{equation}

With this new definition, equation \ref{eqn:backpropagationlayer2} can be rewritten as follows:

\begin{equation}
\begin{split}
     \frac{\partial c}{\partial w^{L_2}} &= \frac{\partial c}{\partial a^{L_2}} \cdot \frac{\partial a^{L_2}}{\partial z^{L_2}} \cdot \frac{\partial z^{L_2}}{\partial w^{L_2}} \\ &= \delta^{L_2} \cdot \frac{\partial z^{L_2}}{\partial w^{L_2}} = \delta^{L_2} \cdot a^{L_1}
\end{split}
\label{eqn:backpropagation_b}
\end{equation}

\begin{equation}
\begin{split}
     \frac{\partial c}{\partial b^{L_2}} &= \frac{\partial c}{\partial a^{L_2}} \cdot \frac{\partial a^{L_2}}{\partial z^{L_2}} \cdot \frac{\partial z^{L_2}}{\partial b^{L_2}} \\ &= \delta^{L_2} \cdot \frac{\partial z^{L_2}}{\partial b^{L_2}} = \delta^{L_2} \cdot 1
\end{split}
\label{eqn:backpropagation_w}
\end{equation}

But the model does not only depend on the second layer and $W_2$, it also depends on the first layer and $W_1$. This is where the beauty of the backpropagation algorithm comes in. The error can be backpropagated in a similar way using the same reasoning. First the composition of functions is shown:

\begin{equation}
\begin{split}
    \frac{\partial c}{\partial w^{L_1}} &= \frac{\partial c(a(z^{L_2}(a(z^{L_1}))))}{\partial w^{L_2}} \\
    \frac{\partial c}{\partial b^{L_1}} &= \frac{\partial c(a(z^{L_2}(a(z^{L_1}))))}{\partial b^{L_2}} 
    \label{eqn:backpropagationsimplenetworklayer2}
\end{split}
\end{equation}

Using the chain rule it is divided into different parts:

\begin{equation}
\begin{split}
     \frac{\partial c}{\partial w^{L_1}} &= \frac{\partial c}{\partial a^{L_2}} \cdot \frac{\partial a^{L_2}}{\partial z^{L_2}} \cdot \frac{\partial z^{L_2}}{\partial a^{L_1}} \cdot \frac{\partial a^{L_1}}{\partial z^{L_1}} \cdot \frac{\partial z^{L_1}}{\partial x} \\
     \frac{\partial c}{\partial b^{L_1}} &= \frac{\partial c}{\partial a^{L_2}} \cdot \frac{\partial a^{L_2}}{\partial z^{L_2}} \cdot \frac{\partial z^{L_2}}{\partial a^{L_1}} \cdot \frac{\partial a^{L_1}}{\partial z^{L_1}} \cdot \frac{\partial z^{L_1}}{\partial 1} \\
\end{split}
 \label{eqn:backpropagationlayer1}
\end{equation}

In fact, of the $6$ derivatives shown, only one would need to be calculated. The derivatives $\frac{\partial c}{\partial a^{L_2}} \cdot \frac{\partial a^{L_2}}{\partial z^{L_2}} $ have already been calculated and are equal to $\delta^{L_2}$. As for the derivative $\frac{\partial z^{L_1}}{\partial x}$ If you want to use the first layer, simply get the result of the previous layer or use $x$ if it is the first layer as it is in this case. The derivative $\frac{\partial z^{L_1}}{\partial 1}$ is a constant value so nothing happens. Finally, the derivative $\frac{\partial a^{L_1}}{\partial z^{L_1}}$ is simply the derivative of the activation function of that layer. This is why in section \ref{activationfunction} it was explained that it is better to use a single activation function for the whole layer and thus facilitate the calculations.
\newline

The only derivative that remains to be explained is $\frac{\partial z^{L_2}}{\partial a^{L_1}}$ the one that expresses how the weighted sum of a layer varies when the input vector of that layer is varied. The calculation of this derivative is simply the matrix of parameters $W^{L_1}$ that connects both layers. Basically, this derivative moves the error from layer $L_2$ to layer $L_1$ distributing the error according to the weightings of the connections. As it was done with the previous layer, with this layer it is also possible to define the imputed error:


\begin{equation}
   \frac{\partial c}{\partial a^{L_1}} = \frac{\partial c}{\partial a^{L_2}} \cdot \frac{\partial a^{L_2}}{\partial z^{L_2}} \cdot \frac{\partial z^{L_2}}{\partial a^{L_1}} \cdot \frac{\partial a^{L_1}}{\partial z^{L_1}} = \delta^{L_1}
\end{equation}

Therefore, equation \ref{eqn:backpropagationlayer1} would look like this:

\begin{equation}
\begin{split}
     \frac{\partial c}{\partial w^{L_1}} &= \delta^{L_1} \cdot \frac{\partial z^{L_1}}{\partial x} \\
     \frac{\partial c}{\partial b^{L_1}} &= \delta^{L_1} \frac{\partial z^{L_1}}{\partial 1} \\
\end{split}
\end{equation}

In summary, it has been possible to explain how to propagate the error to any layer of the network and calculate the derivative with respect to the parameters with the following equations:

\begin{equation}
\begin{split}
     \frac{\partial c}{\partial w^{L_1}} &= \delta^{L_1} \cdot \frac{\partial z^{L_1}}{\partial x} \\
     \frac{\partial c}{\partial b^{L_1}} &= \delta^{L_1} \\
     \frac{\partial c}{\partial w^{L_2}} &= \delta^{L_2} \cdot \frac{\partial z^{L_2}}{\partial a^{L_1}} \\
     \frac{\partial c}{\partial b^{L_2}} &= \delta^{L_2}
\end{split}
\end{equation}

In view of the backpropagation algorithm for the network defined in figure \ref{fig:basic_network}, the algorithm is divided into different steps:
\begin{enumerate}
    
    \item Initialise index to know in the layer to be computed
    \begin{equation}
    \begin{split}
        i=l; l > 0 \\
    \end{split}
    \end{equation}
    
    
    \item \label{alg:backpropagation_loop} Calculation of the imputed error of the layer:
    \begin{enumerate}
        \item If $i=l$:
        \begin{equation}
        \delta^{L_i} = \frac{\partial c}{\partial a^{L_i}} \cdot \frac{\partial a^{L_i}}{\partial z^{L_i}}
        \end{equation}
        
        \item \acrshort{ioc}:
        
        \begin{equation}
        \delta^{L_i} = \delta^{L_{i+1}} \cdot \frac{z^{L_{i+1}}}{\partial a^{L_i}} \cdot \frac{\partial a^{L_i}}{\partial z^{L_i}}
        \end{equation}
    \end{enumerate}
    
    \item 3.	Calculation of derivatives on parameters $b$ and $W$:
    \begin{enumerate}
        \item Calculation on $b$:
        \begin{equation}
        \frac{\partial c}{\partial b^{L_i}} = \delta^{L_{i}}
        \end{equation}
        
        \item Calculation on $w$:
        \begin{enumerate}
            \item If $i=1$:
                \begin{equation}
                \frac{\partial c}{\partial w^{L_i}} = \delta^{L_{i}} \cdot x
                \end{equation}
            \item e.o.c:
                \begin{equation}
                \frac{\partial c}{\partial w^{L_i}} = \delta^{L_{i}} \cdot a^{L_{i-1}}
                \end{equation}
        \end{enumerate}
    \end{enumerate}
    
    \item Update the index:
    \begin{equation}
    i = i-1
    \end{equation}
    
    \item If $i>0$ go back to step \ref{alg:backpropagation_loop}
\end{enumerate}

With this, it has been possible to obtain the partial derivatives of each layer of the network so that it will be possible to calculate the new $W$ matrices using the method of the descent of the gradient explained in section \ref{minimizing-error}.
\subsubsection{Tasa de aprendizaje}\label{learningrate}
Seleccionar una tasa de aprendizaje adecuada para el modelo es un paso fundamental a la hora de diseñar una red neuronal y una mínima modificación de este valor puede tener un gran impacto en el modelo final. 
\newline

Si se selecciona una tasa de aprendizaje muy pequeña, significa que no se fía del resultado del gradiente y por lo tanto en cada iteración el cambio que sufrirá la matriz $W$ será pequeño y el algoritmo se podrá atascar en alguno de los puntos locales mínimos porque el cambio entre la $W$ antigua y la nueva $W$ no está siendo tan drástico como debería para no acabar en estos mínimos locales. Por el contrario, si se selecciona una tasa de aprendizaje muy grande, el algoritmo se sobrepasará por completo y divergirá.
\newline

Por lo tanto, el valor para la tasa de aprendizaje no debe de ser ni muy pequeño para que el algoritmo no se atasque ni muy grande para que el modelo pueda convergir. Una forma de elegir una buena tasa de aprendizaje es probar varios valores y estudiar qué valor funciona mejor. Este algoritmo es conocido como SGD \cite{kiefer}, pero otra opción es usar algún optimizador.
\newline
\subsubsection{Optimizadores}
Los optimizadores son algoritmos que se usan para calcular una tasa de aprendizaje de forma dinámica, no es un valor estático, sino que varía en función del estado del entrenamiento. Según vaya ejecutando el entrenamiento, puede que la tasa de aprendizaje incremente o puede que decremente.
\newline

Si usamos la metáfora de la persona en una montaña, el número de pasos que va a bajar esa persona estará condicionada a distintos criterios. Algunos de esos criterios pueden ser: Cuanto se ha bajado últimamente o como de rápido se ha bajado. 
\newline

Es decir, la tasa de aprendizaje que se use se irá adaptando en función a distintos criterios. Uno de esos criterios es como de rápido el aprendizaje está yendo respecto a la pérdida de nuestro modelo entre otras.
\newline

% The Optimizer - Stochastic Gradient Descent¶ - https://www.kaggle.com/ryanholbrook/stochastic-gradient-descent


Dependiendo del optimizador escogido, usarán unas u otros criterios para ir modificando este valor.  Los optimizadores más usados son: Adam\cite{kingma}, Adadelta\cite{zeiler}, Adagra\cite{duchi} o \acrshort{rmsprop}\cite{duchi}.
\newline
\subsubsection{\textit{Overfitting}}\label{overfitting}

Sabiendo el funcionamiento básico de una red neuronal usando el algoritmo de \textit{forward-pass} y \textit{backpropagation} se puede pensar que si en cada iteración del modelo, la red entrena y aprende nuevos patrones, se puede sacar la conclusión de que realizando un entrenamiento infinito se puede obtener un modelo perfecto cuyo el error se ha minimizado lo máximo posible y por lo tanto los resultados calculados son los esperados. Pero esto no es así, tener una configuración errónea de la red o entrenar a la red más tiempo de lo necesario provoca lo que es conocido como \textit{overfitting}.
\newline

El \textit{overfitting} es un problema bien conocido a la hora de entrenar redes neuronales. En español, significa sobreajustado y suele ser causa de usar un entrenamiento que se ha realizado por mucho tiempo, provocando que el modelo se ajuste perfectamente a los datos de entrenamiento y de algún modo “memorice” los datos y por lo tanto, no sepa extrapolar y adaptarse a otros casos. En otras palabras, se sobreajustan la matrices $W$ de todas las capas y así, el modelo se adapta perfectamente a los datos de entrada produciendo que el modelo no sea capaz de estimar un valor correcto cuando se usa un vector de entrada que antes no lo había visto.

\begin{figure}[H]
    \centering
    \includegraphics[width=12cm]{images/state-of-art/overfitting/overfitting.png}
    \caption{Ejemplo de \textit{overfitting} en una regresión lineal.}
    \label{fig:basic_network}
\end{figure}

manteniendo la precisión de los datos de prueba mientras nuestra red se entrena. 

Una de las maneras más rápidas de saber si un modelo se está entrenando con \textit{overfitting} es usando un segundo dataset además del de entrenamiento. Normalmente, antes de crear una red neuronal hay que realizar un preprocesado de datos explicadas en la Sección TODO. Como bien se explica en esa sección, el dataset original se suele dividir en tres dataset distintos: entrenamiento, validación y test. Obviamente, el \textit{dataset} de entrenamiento es usado para entrenar e ir ajustando los parámetros de las matrices $W$ de forma iterativa. Por otro lado, se tiene los \textit{datasets} de validación y test que son usados para evaluar distintas métricas. Si el problema que se quiere resolver con la red es de tipo regresión se suele usar como métrica algún tipo de error explicados en la Sección \ref{costfunction} como por ejemplo: \acrshort{mae}, \acrshort{mse} o error de \textit{Huber}. Por otro lado si el problema a resolver es de clasificación, se suele usar la precisión como métrica.
\newline

Si vemos que la precisión de los datos de prueba ya no mejora, entonces deberíamos dejar de entrenar. Por supuesto, en sentido estricto, esto no es necesariamente un signo de \textit{overfitting}. Podría ser que la precisión de los datos de la prueba y los datos de entrenamiento dejen de mejorar al mismo tiempo. Aún así, la adopción de esta estrategia evitará el exceso de adaptación.
\newline

Al finalizar de cada \textit{epoch}, se usarán un subconjunto de datos del \textit{dataset} de entrenamiento y otro subconjunto del \textit{dataset} de validación. Con ellos, se calculará el valor de la métrica seleccionada y se obtendrán dos valores. Estos valores se puede comparar entre ellos. Dos valores que son parecidos indican que el modelo es capaz de estrapolar a otros casos que no ha usado para el entrenamiento. Por el contrario, si el valor asociado al \textit{dataset} de entrenamiento es mucho mejor que el asociado al de validación esto puede significar que el modelo pueda estar sufriendo de \textit{overfitting}. Un entrenamiento llevado a cabo sin \textit{overfitting} se puede visualizar en la Figura \ref{fig:overfitting}.
\newline

Al inicio del entrenamiento, como la red no ha sido entrenada y los paramétros inicializados aleatoriamente, el modelo tendrá unas métricas bastantes malos, es decir, si se está usando algún error, dicho valor será muy alto. Por otro lado si se está midiendo la precisión del modelo dicho valor distará mucho del 100\%. Estas métricas serán igualmente malas tanto para ambos \textit{datasets}. Según vayan ocurriendo \textit{epochs} en el entrenamiento, llegará un momento que el modelo no pueda mejorar más y comenzará a memorizar los datos con los que esta siendo entrenado del \textit{dataset} de entrenamiento provocando que la métrica asociada al entrenamiento sea casi perfecta y la de validación progresivamente vaya empeorando. Un ejemplo de un entremiento que sufre de \textit{overfitting} puede verse de la siguiente manera:
\newline

\begin{figure}[H]
    \centering
    \includegraphics[width=12cm]{images/state-of-art/overfitting/overfitting-loss.png}
    \caption{Ejemplo de \textit{overfitting} en una regresión lineal usando un error como métrica.}
    \label{fig:overfitting}
\end{figure}

Al comienzo del entrenamiento ambas métricas van en pararelo, hasta que llegan a la \textit{epoch} 10 aproximadamente, en ese punto se puede visualizar como la métrica de validación poco a poco va empeorando. Es en ese punto donde se ve que la red tiene un problema de \textit{overfitting}. Dependiendo de la métrica usada, el \textit{overfitting} puede aparecer antes o después, pero es un tema más complejo que no se explicará en este trabajo.
\newline

Aún así hay varias técnicas que se explicarán a continuación para evitar el \textit{overfitting}: 
\begin{itemize}
    \item Regularización L1 y L2:
    Son dos métodos que calculan un valor conocido como penalización que se añade al error y de ese modo poder penalizar parámetros con valores grandes. Si una neurona tiene valores grandes puede ser signo de que la neurona está intentando memorizar. De hecho, se considera una mala práctica dejar que un conjunto pequeño de neuronas de la red sean las que más responsabilidad tengan, es decir, que sus parámetros asociados sean muy grandes.
    \newline
    
    Por un lado, se tiene L1 que es la suma de todos los valores absolutos de $w$ y $b$. Es una penalización linear ya que la función asociada es directamente proporcional a los parámetros. La penalización L2 es la suma de todos los parámetros $w$ y $b$ al cuadrado. Es una función no lineal y penaliza con mayor intensidad a os valores grande a diferencia de L1 que penaliza mucho más en proporción a los valores pequeños por ser lineal. Esto causa que el modelo empieza a ser invariante a pequeños valores de entrada y variante solo a los valores grandes. Por ello, L1 es raramente usado a no ser que sea en combinación con L2. Ambas funciones están en función de un valor $\lambda$, con este valor se puede dictar cuanto impacto tendrá L1 y L2 en el error final. Matemáticamente:
    
    \begin{align*}
         L_{1w} = \lambda \sum_m |w_m| &&  L_{2w} = \lambda \sum_m w^2_m \addtocounter{equation}{1}\tag{\theequation} \\ 
         L_{1b} =\lambda \sum_n |b_n| && L_{2b} = \lambda \sum_n b^2_n \addtocounter{equation}{1}\tag{\theequation}
    \end{align*}
    

    El error final viene dado por la simple suma del error calculado por la función de coste y L1 con L2:
    \begin{equation}
        c_T = c + L_{1w} + L_{1b} + L_{2w} + L_{2b}
    \end{equation}
    
    Como se usa la regularización para el calculo de $c$, hace falta saber su derivada para poder usar el \textit{back-propagation}. Quedando la derivada de coste de la siguiente forma:
    
    \begin{equation}
        \frac{\partial c_T}{\partial w^{L_i}} = c' + L'_{1w} + L'_{1b} + L'_{2w} + L'_{2b}
    \end{equation}
    
    Las derivadas de L1 y L2 son las siguientes:
    \begin{align*}
         L'_{1w} = \lambda \begin{cases} 1,& \text{si } w_m > 1\\ -1,& \text{si } w_m < 1\end{cases} && L'_{2w} =  2\lambda w_m \addtocounter{equation}{1}\tag{\theequation} \\ 
         L'_{1b} =  \lambda \begin{cases} 1,& \text{si } b_n > 1\\ -1,& \text{si } b_n < 1\end{cases} && L'_{2b} = 2\lambda b_n \addtocounter{equation}{1}\tag{\theequation}
    \end{align*}
    
    \item \textit{Dropout} \label{dropout}: Usando esta técnica, en cada iteración del entrenamiento la red se modifica. Supongamos que se tiene un \textit{input} $x$ y el valor deseado $y$. Ordinariamente, se entrenaría mediante la propagación hacia adelante de $x$ a través de la red, y luego la propagación hacia atrás para determinar la contribución al gradiente. Con \textit{dropout}, este proceso se modifica. Se comienza eliminando aleatoriamente (y temporalmente) un porcentaje de las neuronas ocultas en la red, mientras que se deja las neuronas de entrada y salida sin modificar. Las neuronas que han sido borradas, serán neuronas "fantasmas" para ese \textit{epoch}:
    
    \begin{figure}[H]
        \centering
        \includegraphics[width=12cm]{images/state-of-art/overfitting/dropout-1.png}
        \caption{\textit{Dropout} aplicado a una red con un $50\%$  de probabilidades}
        \label{fig:basic_network}
    \end{figure}
    
    Una vez terminado el proceso de propagación hacia adelante y \textit{backpropagation}, se comenzará un nuevo \textit{epoch} eliminando aleatoriamente un conjunto de neuronas. Es resumen, por cada iteración, se eliminar un subconjunto de las neuronas de acuerdo a un porcentaje dado.
    \newline
    
    Repitiendo este proceso durante toda la fase de entrenamiento, la red aprenderá unas matrices $W$ que se habrán aprendido en condiciones en las que un subconjunto de las neuronas ocultas fueron eliminadas. Cuando realmente se ejecuta la red completa, significará que el número de neuronas activas será mayor. Por ello, se compensar reduciendo la parte proporcional con las que fueron entradas. Por ejemplo, usando un porcentaje igual a $50$, los valores de $W$ serán la mitad de lo que el gradiente haya calculado.
    
    \item Selección de una tasa de aprendizaje de forma iterativa: Esta técnica se basa en el estudio de varias tasas de aprendizaje en función del \textit{epoch} del entrenamiento. Se suele dar un conjunto de valores de forma ordenada entre un mínimo y un máximo. Cada valor de este intervalo será una tasa de aprendizaje para distintas \textit{epochs} y se podrá estudiar su comportamiento y ver si el descenso del gradiente es capaz de converger. 
    \newline
    
    Si es capaz de converger, los resultados obtenidos serán los que se esperan, haciendo que la función de coste se reduzca de forma progresiva. Por el contrario, si la tasa de aprendizaje es muy baja o muy alta, provocará que el descenso del gradiente no pueda converger resultando en que el resultado de la función de coste sea irregular y por lo general empeorando el modelo.

\end{itemize}


