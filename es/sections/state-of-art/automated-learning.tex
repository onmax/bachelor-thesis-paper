\subsection{Aprendizaje automático}

El aprendizaje automático (\acrfull{ml} en inglés) es la rama de la \acrshort{ai} que estudia como dotar a las máquinas de capacidad de aprendizaje entendiendo a éste como la generalización del conocimiento a partir de un conjunto de experiencias. Este aprendizaje puede dividirse en \cite{amodernapproach}:
\begin{itemize}
\item Supervisado: El algoritmo aprende en un conjunto de datos etiquetados, proporcionando una clave de respuesta que el algoritmo puede utilizar para evaluar su precisión en los datos de entrenamiento \cite{nvidia}.
\item No supervisado: Dados los datos no etiquetados, el algoritmo trata de captar sentido extrayendo características y patrones por sí mismo
\item Reforzado: Se entrena un algoritmo con un sistema de recompensas, proporcionando retroalimentación cuando un agente de inteligencia artificial realiza la mejor acción en una situación particular \cite{nvidia}. \end{itemize}

El aprendizaje automático es la rama principal de la \acrshort{ai}, ya que el resto de las categorías o bien imitan comportamientos o bien aprenden de experiencias, es decir, usan \acrfull{ml} para realizar la tarea que se les encomienda. Una cosa es programar una máquina para que pueda moverse y otra cosa es que la máquina aprenda a moverse. Igualmente, no es lo mismo programar que componentes forman una cara de una persona que automáticamente aprender que es una cara. Este cambio de paradigma es lo que diferencia el \acrshort{ml} de la \acrshort{ai}, y es por ello por lo que no se debe pensar que son los mismos conceptos.
\newline

El aprendizaje automático es el estudio sistemático de algoritmos y sistemas que mejoran su conocimiento o desempeño con experiencia. En esta subcategoría existen diferentes tipos de aplicaciones siendo algunos de ellos \cite{amodernapproach}:
\begin{itemize}
\item Modelos de regresión: Son modelos que predicen el valor de una o más variables dado un vector de entrada. Una función lineal es el modelo de regresión más simple, pero los modelos de regresión más complejos están formados por un conjunto fijo de funciones no lineales, conocidas como funciones base \cite{flach}. Algoritmos como \acrfull{svm} o las \acrfull{nn} se basan en este tipo de modelos. Este último, será el usado en el presente trabajo.
\item Árboles de decisión
\item Modelos de clasificación
\item Técnicas de agrupación
\end{itemize}

\subsubsection{Models}\label{models}

The universe we know is in constant evolution and is complex, chaotic and enigmatic, however, the intelligence of the human being manages to give meaning to all that chaos in a search for the elegance and symmetry that is hidden among the patterns it identifies in our reality. The ability to detect patterns and use them for our own good has been one of the main reasons for the development of the human species. Science has allowed us to understand, observe and simplify the world by turning all this enigma into knowledge, that is, by reconstructing reality through models \cite{marlie}.
\newline

A model is a conceptual and simplified construction of a complex reality allowing a better understanding of that reality. There are many models that we use every day, for example, a map. A map allows us to reflect a three-dimensional world on a two-dimensional surface, eliminating information that we do not need to process, such as environmental artifacts or types of vegetation. Another example is a physical equation, where different constants and values are related and in this way, we can approximate the physical behaviour of reality. A score is another example of a model. It reflects information on how different instruments must be coordinated in order to always produce the same song. The frequency spectrum of the song could be used to better represent reality, but this is obviously much more complex to interpret as a human being. In short, a model seeks a balance between correctly representing reality and being simple so that it can be used \cite{kuhne}.
\newline

Let's imagine that you want to model the weather. For this purpose, different pieces of evidence are collected, and, after observation, a first model can be made:

\begin{displayquote}
“The summer will be sunny, hot and clear. The winter will be cold, cloudy, and rainy.”
\end{displayquote}

If you keep collecting evidence, you will soon realise that this model is very simple as there will be days that are cold in summer and hot in winter. There will be storms in summer and clear days in winter. This will lead to improvements in the model in each iteration.
\newline


But if you keep studying other evidence in some tropics or in some poles, for example, you will conclude that the model was very simple and therefore more rules will have to be added to make this model more similar to reality anywhere in the world.
\newline

In the end, you will get a very complex model with all the exceptions and conditions. An alternative to this is to make use of probability to be able to say mathematically that most of the time a summer day will be sunny and not such a complex model to depend on.
\newline

Probability is the perfect tool to limit the uncertainty about a subject due to lack of knowledge or data or to avoid the work of making a complex model that would lead to less understanding for the human mind and dispersion in its approach. Being able to use a probability is much easier than having to study all the physical conditions of an environment and the behaviour of its entities in order to know for sure what is going to happen. Using probability to build models results in probabilistic models. These models compress many of the variability of our reality based on probabilities, making it easier to manage the information we receive from the environment \cite{kuhne}.
\newline

Our brain applies similar schemes to these probabilistic models and it is thanks to them that we have the ability to conceptualise, predict, generalise, reason or learn. For this reason, discovering what these models are is one of the basic objectives of the field of \acrlong{ml} and one of the fundamental tools of \acrshort{ai}.
\newline