\subsubsection{Ventanas deslizantes} \label{sliding-windows}

Una ventana deslizante es un método que se usa para entrenar modelos para predicción de variables a lo largo del tiempo. Como se explicaba en la sección \ref{window-generator}, un generador de ventana es un proceso en el cual dadas dos variables, se obtendrá una ventana que contendrá: un conjunto de intervalos de entrada y un conjunto de intervalos de salida. El tamaño de cada uno de los conjuntos son las variables de la ventana (\textit{input_width} y \textit{label_width}). Bien, pues una ventana deslizante es un método por el cual se entrena una red con la primera ventana. En la segunda \textit{epoch}, se entrena la red con una segunda ventana que es igual que la anterior pero deslizando la ventana un intervalo en el futuro.
\newline

Gráficamente, una ventana deslizante se puede ver de la siguiente forma:

% TODO Gráfica-


En este trabajo se han definido varias ventanas para cada uno de los modelos, para poder comparar todos ellos y estudiar la mejor combinación de variables para las variables de ventanas deslizantes. En total, se han generado las siguientes ventanas deslizantes con las siguientes variables:

% TODO Hacer tabla